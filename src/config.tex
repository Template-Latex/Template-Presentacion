% Template:     Presentación LaTeX
% Documento:    Configuraciones del template
% Versión:      1.3.4 (08/09/2021)
% Codificación: UTF-8
%
% Autor: Pablo Pizarro R.
%        pablo@ppizarror.com
%
% Manual template: [https://latex.ppizarror.com/presentacion]
% Licencia MIT:    [https://opensource.org/licenses/MIT]

% CONFIGURACIONES GENERALES
\def\compilertype {pdf2latex}      % Compilador {pdf2latex,xelatex,lualatex}
\def\documentfontsize {10}         % Tamaño de la fuente del documento [pt]
\def\documentinterline {1}         % Interlineado del documento [factor]
\def\fontdocument {lmodern}        % Tipografía base, ver soportadas en manual
\def\fonttypewriter {tmodern}      % Tipografía de \texttt, ver manual
\def\fonturl {same}                % Tipo de fuente url {tt,sf,rm,same}
\def\frametextjustified {true}     % Justifica todos los párrafos de los frames
\def\graphicxdraft {false}         % En true no carga las imágenes (modo draft)
\def\pointdecimal {true}           % N° decimales con punto en vez de coma
\def\showlayoutlines {false}       % Muestra el layout de la página
\def\useenglishbabel {false}       % Inglés, desactivar para otros idiomas
\def\usespanishbabel {true}        % Español, desactivar para otros idiomas

% CONFIGURACIÓN DE LAS LEYENDAS - CAPTION
\def\captionalignment {justified}  % Posición {centered,justified,left,right}
\def\captioncolor {black}          % Color nombre objeto (código,figura,tabla)
\def\captionfontsize{footnotesize} % Tamaño de fuente de los caption
\def\captionlabelformat {simple}   % Formato leyenda {empty,simple,parens}
\def\captionlabelsep {colon}       % Sep. {none,colon,period,space,quad,newline}
\def\captionlessmarginimage {0.1}  % Margen sup/inf de figura si no hay ley. [cm]
\def\captionlrmargin {0}           % Márgenes izq/der de la leyenda [cm]
\def\captionlrmarginmc {0}         % Margen izq/der leyenda dentro de columnas [cm]
\def\captionmarginimage {0}        % Margen vertical entre caption e imagen [cm]
\def\captionmarginimages {-0.04}   % Margen v entre caption y entorno images [cm]
\def\captionmarginimagesmc {-0.04} % Margen v entre caption y entorno imagesmc [cm]
\def\captionmarginmultimg {0}      % Margen izq/der leyendas múltiple img [cm]
\def\captionnumcode {arabic}       % N° código {arabic,alph,Alph,roman,Roman}
\def\captionnumequation {arabic}   % N° ecuaciones {arabic,alph,Alph,roman,Roman}
\def\captionnumfigure {arabic}     % N° figuras {arabic,alph,Alph,roman,Roman}
\def\captionnumsubfigure {alph}    % N° subfiguras {arabic,alph,Alph,roman,Roman}
\def\captionnumsubtable {alph}     % N° subtabla {arabic,alph,Alph,roman,Roman}
\def\captionnumtable {arabic}      % N° tabla {arabic,alph,Alph,roman,Roman}
\def\captionsubchar {.}            % Caracter entre N° objeto y N° subfigura/tabla
\def\captiontbmarginfigure {9.35}  % Margen sup/inf de la leyenda en figuras [pt]
\def\captiontbmargintable {7}      % Margen sup/inf de la leyenda en tablas [pt]
\def\captiontextbold {true}        % Etiqueta (código,figura,tabla) en negrita
\def\captiontextcolor {black}      % Color de la leyenda
\def\captiontextsubnumbold {true}  % N° subfigura/subtabla en negrita
\def\codecaptiontop {true}         % Leyenda arriba del código fuente
\def\equationcaptioncenter {true}  % Ecuaciones están centradas o justificadas
\def\figurecaptiontop {false}      % Leyenda arriba de las imágenes
\def\marginaligncaptbottom {0.1}   % Margen inferior caption en align [cm]
\def\marginaligncapttop {-0.75}    % Margen superior caption en align [cm]
\def\marginalignedcaptbottom {0.1} % Margen inferior caption en aligned [cm]
\def\marginalignedcapttop {-0.75}  % Margen superior caption en aligned [cm]
\def\margineqncaptionbottom {0}    % Margen inferior caption ecuación [cm]
\def\margineqncaptiontop {-0.7}    % Margen superior caption ecuación [cm]
\def\margingathercaptbottom {0.1}  % Margen inferior caption en gather [cm]
\def\margingathercapttop {-0.9}    % Margen superior caption en gather [cm]
\def\margingatheredcaptbottom{0.1} % Margen inferior caption en gathered [cm]
\def\margingatheredcapttop {-0.7}  % Margen superior caption en gathered [cm]
\def\sectioncaptiondelimiter {.}   % Caracter delimitador n° objeto y sección
\def\showsectioncaptioncode {none} % N° sec. código {none,chap,(s/ss/sss/ssss)ec}
\def\showsectioncaptioneqn {none}  % N° sec. ecuación {none,chap,(s/ss/sss/ssss)ec}
\def\showsectioncaptionfig {none}  % N° sec. figuras {none,chap,(s/ss/sss/ssss)ec}
\def\showsectioncaptionmat {none}  % N° matemático {none,chap,(s/ss/sss/ssss)ec}
\def\showsectioncaptiontab {none}  % N° sec. tablas {none,chap,(s/ss/sss/ssss)ec}
\def\subcaptionfsize{footnotesize} % Tamaño de la fuente de los subcaption
\def\subcaptionlabelformat{parens} % Formato leyenda sub. {empty,simple,parens}
\def\subcaptionlabelsep {space}    % Sep. {none,colon,period,space,quad,newline}
\def\tablecaptiontop {true}        % Leyenda arriba de las tablas

% ANEXO, CITAS, REFERENCIAS
\def\bibtexrefsep {6}              % Separación entre refs. {bibtex} [pt]
\def\bibtexstyle {apalike}         % Formato referencias bibtex {apa,ieeetr,etc...}
\def\stylecitereferences {bibtex}  % Estilo cita/ref {bibtex,custom}

% CONFIGURACIONES DE OBJETOS
\def\columnsepwidth {2.1}          % Separación entre columnas [em]
\def\defaultimagefolder {img/}     % Carpeta raíz de las imágenes
\def\equationleftalign {false}     % Ecuaciones alineadas a la izquierda
\def\equationrestart {none}        % Reinicio n° {none,chap,(s/ss/sss/ssss)ec}
\def\footnotetwocolumn {false}     % Footnote en dos columnas
\def\footnotepagetoprule {false}   % Footnote en pag. tienen separador superior
\def\footnoterestart {none}        % N° footnote {none,chap,page,(s/ss/sss/ssss)ec}
\def\fpremovetopbottomcenter{true} % Elimina espacio vertical al centrar con b!,t!
\def\imagedefaultplacement {H}     % Posición por defecto de las imágenes
\def\marginalignbottom {-0.4}      % Margen inferior entorno align [cm]
\def\marginalignedbottom {-0.2}    % Margen inferior entorno aligned [cm]
\def\marginalignedtop {-0.4}       % Margen superior entorno aligned [cm]
\def\marginaligntop {-0.4}         % Margen superior entorno align [cm]
\def\marginequationbottom {-0.2}   % Margen inferior ecuaciones [cm]
\def\marginequationtop {0}         % Margen superior ecuaciones [cm]
\def\marginfloatimages {-13}       % Margen sup. figuras insertimageleft/right [pt]
\def\marginfootnote {10}           % Margen derecho footnote [pt]
\def\margingatherbottom {-0.2}     % Margen inferior entorno gather [cm]
\def\margingatheredbottom {-0.1}   % Margen inferior entorno gathered [cm]
\def\margingatheredtop {-0.4}      % Margen superior entorno gathered [cm]
\def\margingathertop {-0.4}        % Margen superior entorno gather [cm]
\def\marginimagebottom {-0.50}     % Margen inferior figura [cm]
\def\marginimagemultbottom {-0.05} % Margen inferior imágenes múltiples [cm]
\def\marginimagemultright {0.35}   % Margen derecho imágenes múltiples [cm]
\def\marginimagemulttop {-0.3}     % Margen superior imágenes múltiples [cm]
\def\marginimagetop {0}            % Margen superior figuras [cm]
\def\numberedequation {true}       % Ecuaciones con \insert... numeradas
\def\senumerti {\arabic{enumi}.}   % Estilo enumerate nivel 1
\def\senumertii {\alph{enumii})}   % Estilo enumerate nivel 2
\def\senumertiii{\roman{enumiii}.} % Estilo enumerate nivel 3
\def\senumertiv {\Alph{enumiv})}   % Estilo enumerate nivel 4
\def\sitemizei {\iitembcirc}       % Estilo itemize nivel 1
\def\sitemizeii {\iitemdash}       % Estilo itemize nivel 2
\def\sitemizeiii {\iitemcirc}      % Estilo itemize nivel 3
\def\sitemizeiv {\iitembsquare}    % Estilo itemize nivel 4
\def\sourcecodefontf {\ttfamily}   % Tipo de letra código fuente
\def\sourcecodefonts {\small}      % Tamaño letra código fuente
\def\sourcecodenumbersep {3}       % Separación entre número línea y código [pt]
\def\sourcecodetabsize {3}         % Tamaño tabulación código fuente
\def\tabledefaultplacement {H}     % Posición por defecto de las tablas
\def\tablepaddingh {0.75}          % Espaciado horizontal de celda de las tablas
\def\tablepaddingv {1.15}          % Espaciado vertical de celda de las tablas
\def\tikzdefaultplacement {H}      % Posición por defecto de las figuras tikz

% CONFIGURACIÓN DE LOS COLORES DEL DOCUMENTO
\def\highlightcolor {yellow}       % Color del subrayado con \hl
\def\linenumbercolor {gray}        % Color del n° de línea (\showlinenumbers)
\def\linkcolor {black}             % Color de los links del documento
\def\maintextcolor {black}         % Color principal del texto
\def\numcitecolor {black}          % Color del n° de las referencias o citas
\def\showborderonlinks {false}     % Color de un link por un recuadro de color
\def\sourcecodebgcolor {lgray}     % Color de fondo del código fuente
\def\tablelinecolor {black}        % Color de las líneas de las tablas
\def\tablerowfirstcolor {none}     % Primer color de celda de las tablas
\def\tablerowsecondcolor {gray!20} % Segundo color de celda de las tablas
\def\urlcolor {magenta}            % Color de los enlaces web (\href,\url)

% OPCIONES DEL PDF COMPILADO
\def\cfgbookmarksopenlevel {1}     % Nivel marcadores en pdf (1:secciones)
\def\cfgpdfbookmarkopen {true}     % Expande marcadores del nivel configurado
\def\cfgpdfcenterwindow {true}     % Centra ventana del lector al abrir el pdf
\def\cfgpdfcopyright {}            % Establece el copyright del documento
\def\cfgpdfdisplaydoctitle {true}  % Muestra título del informe en visor
\def\cfgpdffitwindow {true}        % Ajusta la ventana del lector tamaño pdf
\def\cfgpdfkeywords {}             % Palabras clave del pdf
\def\cfgpdflayout {OneColumn}      % Modo de página {OneColumn,SinglePage}
\def\cfgpdfmenubar {true}          % Muestra el menú del lector
\def\cfgpdfpageview {FitBV}        % {Fit,FitH,FitV,FitR,FitB,FitBH,FitBV}
\def\cfgpdfsecnumbookmarks {true}  % Número de la sec. en marcadores del pdf
\def\cfgpdftoolbar {true}          % Muestra barra de herramientas lector pdf
\def\cfgshowbookmarkmenu {false}   % Muestra menú marcadores al abrir el pdf
\def\indexdepth {4}                % Profundidad de los marcadores
\def\pdfcompilecompression {9}     % Factor de compresión del pdf (0-9)
\def\pdfcompileobjcompression {2}  % Nivel compresión objetos del pdf (0-3)
\def\usepdfmetadata {true}         % Añade metadatos al pdf compilado

% NOMBRE DE OBJETOS
\def\namemathcol {Corolario}       % Nombre de los colorarios
\def\namemathdefn {Definición}     % Nombre de las definiciones
\def\namemathej {Ejemplo}          % Nombre de los ejemplos
\def\namemathlem {Lema}            % Nombre de los lemas
\def\namemathobs {Observación}     % Nombre de las observaciones
\def\namemathprp {Proposición}     % Nombre de las proposiciones
\def\namemaththeorem {Teorema}     % Nombre de los teoremas
\def\namereferences {Referencias}  % Nombre de la sección de referencias
\def\nomltappendixsection {Anexo}  % Etiqueta sección en anexo/apéndices
\def\nomltwfigure {Figura}         % Etiqueta leyenda de las figuras
\def\nomltwsrc {Código}            % Etiqueta leyenda del código fuente
\def\nomltwtable {Tabla}           % Etiqueta leyenda de las tablas

% CONFIGURA BEAMER
% http://tug.ctan.org/macros/latex/contrib/beamer/doc/beameruserguide.pdf
\mode<presentation> {

	% LISTA DE TEMAS GLOBALES, DESCONFIGURAR PARA APLICAR
	% https://deic-web.uab.cat/~iblanes/beamer_gallery/index_by_theme.html

	% \usetheme{AnnArbor}
	% \usetheme{Antibes}
	% \usetheme{Bergen}
	% \usetheme{Berkeley}
	% \usetheme{Berlin}
	% \usetheme{Boadilla}
	% \usetheme{boxes}
	% \usetheme{CambridgeUS}
	% \usetheme{Copenhagen}
	% \usetheme{Darmstadt}
	% \usetheme{default}
	% \usetheme{Dresden}
	% \usetheme{Frankfurt}
	% \usetheme{Goettingen}
	% \usetheme{Hannover}
	% \usetheme{Ilmenau}
	% \usetheme{JuanLesPins}
	% \usetheme{Luebeck}
	\usetheme{Madrid}
	% \usetheme{Malmoe}
	% \usetheme{Marburg}
	% \usetheme{Montpellier}
	% \usetheme{PaloAlto}
	% \usetheme{Pittsburgh}
	% \usetheme{Rochester}
	% \usetheme{Singapore}
	% \usetheme{Szeged}
	% \usetheme{Warsaw}
	
	% TEMAS TIPO <INNER>, USAR SÓLO SI NO SE APLICA EL GLOBAL
	% http://www.cpt.univ-mrs.fr/~masson/latex/Beamer-appearance-cheat-sheet.pdf
	
	% \useinnertheme{default}
	% \useinnertheme{circles}
	%\useinnertheme{rectangles}
	% \useinnertheme{rounded}
	% \useinnertheme{inmargin}
	
	% TEMAS TIPO <OUTER>, USAR SÓLO SI NO SE APLICA EL GLOBAL
	
	% \useoutertheme{default}
	%\useoutertheme{infolines}
	% \useoutertheme{miniframes}
	% \useoutertheme{smoothbars}
	% \useoutertheme{sidebar}
	% \useoutertheme{split}
	% \useoutertheme{shadow}
	% \useoutertheme{tree}
	% \useoutertheme{smoothtree}
	
	% COLORES DE LA SLIDE, USAR SÓLO SI NO SE APLICA EL GLOBAL
	% https://deic-web.uab.cat/~iblanes/beamer_gallery/index_by_color.html
	
	% \usecolortheme{default}
	%\usecolortheme[named=cardinalred]{structure}
	% \usecolortheme{sidebartab}

	% \usecolortheme{albatross}
	% \usecolortheme{beaver}
	% \usecolortheme{beetle}
	% \usecolortheme{crane}
	% \usecolortheme{dolphin}
	% \usecolortheme{dove}
	% \usecolortheme{fly}
	% \usecolortheme{lily}
	% \usecolortheme{monarca}
	%\usecolortheme{orchid}
	% \usecolortheme{rose}
	% \usecolortheme{seagull}
	% \usecolortheme{seahorse}
	% \usecolortheme{spruce}
	%\usecolortheme{whale}
	% \usecolortheme{wolverine}

	% IDIOMA
	\uselanguage{spanish}
	\languagepath{spanish}

	% CONFIGURACIONES GENERALES
	\def\itemizedeleteleftmargin {true}
	\def\opacitytabularframe {0.4}
	
	% CONFIGURA LOS BLOQUES, VALORES EN [EM]
	\def\blockmarginbottom {1}
	\def\blockpaddingbottom {0.6}
	\def\blockpaddingleft {0.6}
	\def\blockpaddingright {0.6}
	\def\blockpaddingtop {0.75}
	
	% PARA RESTABLECER EL HEADLINE, COMENTAR
	\setbeamertemplate{headline}{}
	
	% ELEMENTOS CON \pause o \onslide ESTÁN OCULTOS
	\setbeamercovered{invisible}
	% \setbeamercovered{transparent}
	
	% PARA ELIMINAR EL FOOTER, DESCOMENTAR
	% \setbeamertemplate{footline}{}
	
	% REEMPLAZAR EL FOOTER POR EL N° DE PÁGINA
	% \setbeamertemplate{footline}[page number]
	
	% PARA ACTIVAR LOS SÍMBOLOS DE NAVEGACIÓN, COMENTAR
	\setbeamertemplate{navigation symbols}{}
	
}
