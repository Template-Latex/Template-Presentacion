% Template:     Presentación LaTeX
% Documento:    Archivo de ejemplo
% Versión:      2.1.0 (25/04/2022)
% Codificación: UTF-8
%
% Autor: Pablo Pizarro R.
%        pablo@ppizarror.com
%
% Manual template: [https://latex.ppizarror.com/presentacion]
% Licencia MIT:    [https://opensource.org/licenses/MIT]

% Portada
\begin{frame}
	\titlepage
\end{frame}

% Tabla de contenidos
\begin{frame}
	\frametitle{Contenidos}
	\tableofcontents
\end{frame}

% Organiza el documento en secciones, útiles para la tabla de contenidos
\section{Primera sección}

\subsection{Ejemplo de sub-sección}

%---------------------------------------------------------------------

\begin{frame}{Empezando con el mejor template}
	\lipsum[1] \footnote{Insertar footnotes es muy fácil con el template!}
\end{frame}

%---------------------------------------------------------------------

\begin{frame}{Enumeraciones}
	\begin{itemize}
		\item Lorem ipsum dolor sit amet, consectetur adipiscing elit
		\item Aliquam blandit faucibus nisi, sit amet dapibus enim tempus eu
		\item Nulla commodo, erat quis gravida posuere, elit lacus lobortis est, quis porttitor odio mauris at libero
		\item Nam cursus est eget velit posuere pellentesque
		\item Vestibulum faucibus velit a augue condimentum quis convallis nulla gravida
	\end{itemize}
\end{frame}

%---------------------------------------------------------------------

\begin{frame}{Bloques de texto}
	\begin{block}{Bloque 1}
		Lorem ipsum dolor sit amet, consectetur adipiscing elit. Integer lectus nisl, ultricies in feugiat rutrum, porttitor sit amet augue. Aliquam ut tortor mauris. Sed volutpat ante purus, quis accumsan dolor.
	\end{block}
	
	\begin{alertblock}{Bloque 2 (alerta)}
		Pellentesque sed tellus purus. Class aptent taciti sociosqu ad litora torquent per conubia nostra, per inceptos himenaeos. Vestibulum quis magna at risus dictum tempor eu vitae velit.
	\end{alertblock}
	
	\begin{exampleblock}{Bloque 3 (ejemplo)}
		Suspendisse tincidunt sagittis gravida. Curabitur condimentum, enim sed venenatis rutrum, ipsum neque consectetur orci, sed blandit justo nisi ac lacus.
	\end{exampleblock}
\end{frame}

%---------------------------------------------------------------------

\begin{frame}{Múltiples columnas}
	\begin{columns}[c]
		\column{.45\textwidth} % Columna izquierda y espesor
		\textbf{Heading}
		\begin{enumerate}
			\item Statement
			\item Explanation
			\item Example
		\end{enumerate}
		
		\column{.5\textwidth} % Columna derecha y espesor
		Lorem ipsum dolor sit amet, consectetur adipiscing elit. Integer lectus nisl, ultricies in feugiat rutrum, porttitor sit amet augue. Aliquam ut tortor mauris. Sed volutpat ante purus, quis accumsan dolor.

		\begin{exampleblockjustified}{Bloque justificado (ejemplo)}
			Suspendisse tincidunt sagittis gravida. Curabitur condimentum, enim sed venenatis rutrum, ipsum neque consectetur orci, sed blandit justo nisi ac lacus.
		\end{exampleblockjustified}
	\end{columns}
\end{frame}

%---------------------------------------------------------------------

\begin{frame}{Tabla}
	\enabletablerowcolor[2] % Activa el color de celda
	\begin{table}[H]
		\begin{threeparttable}
		\centering
		\caption{Tabla de cálculo.}
		\begin{tabular}{cccC{4cm}}
			\hline
			\textbf{Elemento} & $\epsilon_i$ & \textbf{Valor} & \textbf{Descripción} \bigstrut \\
			\hline
			A     & 10    & 3,14$\pi$ & Valor muy interesante\tnote{a} \\
			B     & 20    & 6 & Segundo elemento \\
			C     & 30    & 7 & Tercer elemento\tnote{1} \\
			D     & 150    & 10 & Sin descripción \\
			E     & 0    & 0 & Cero \\
			\hline
			\end{tabular}
		\begin{tablenotes}
			\item[a] Este elemento tiene una descripción debajo de la tabla
			\item[1] Más comentarios
		\end{tablenotes}
		\end{threeparttable}
		\label{tab:anexo-1}
	\end{table}
	\disabletablerowcolor % Desactiva el color de celda
\end{frame}

%---------------------------------------------------------------------

\begin{frame}{Figuras}
	La Figura \ref{img:testimage} ilustra una imagen insertada con las funciones propias del template, mismas compartidas con todos los subtemplates.
	
	\insertimage[\label{img:testimage}]{ejemplos/test-image.png}{scale=0.14}{Where are you? de \quotes{Internet}.}
\end{frame}

%---------------------------------------------------------------------

\begin{frame}{Figuras animadas!}
	La Figura \ref{img:testanimado} muestra un ejemplo de figuras animadas. Éste carga imágenes del estilo \texttt{ejemplos/animacion-0} a \texttt{ejemplos/animacion-N} con \texttt{N=19} (parámetro de la función). Notar que esta característica no es posible reproducirla en todos los visualizadores. En Adobe Acrobat se puede visualizar sin problemas $\smiley$.
	
	\insertanimatedimage[\label{img:testanimado}]{ejemplos/animacion}{width=4cm}{10}{19}{Ejemplo animación.}
\end{frame}

%---------------------------------------------------------------------

\begin{frame}{Figuras múltiples}
	\begin{columns}[c]
		\column{.65\textwidth} % Columna izquierda y espesor
		\begin{images}{Ejemplo de imagen múltiple.}
			\addimage{ejemplos/test-image}{width=3cm}{Ciudad}
			\addimageanum{ejemplos/test-image-wrap}{height=2cm}
			\imagesnewline
			\addimage{ejemplos/test-image}{width=4cm}{Ciudad más grande}
		\end{images}
		
		\column{.35\textwidth}
		
		Enumerar con letras:
		
		\begin{enumeratebf}[label=\alph*) ] % Fuente en negrita
			\item Peras
			\item Manzanas
		\end{enumeratebf}
		
		O con números romanos:
		
		\begin{enumerate}[label=\roman*) ]
			\item Rojo
			\item Café
		\end{enumerate}
		
		$\ldots$ o griegos:
		
		\begin{enumerate}[label=\greek*) ]
			\item Matemáticas
			\item Lenguaje
		\end{enumerate}
	\end{columns}
\end{frame}

%---------------------------------------------------------------------

\begin{frame}[fragile]{Código fuente (Template)}
	Al igual que \href{https://latex.ppizarror.com/informe}{Template-Informe}, Template-Presentación ofrece el soporte a todos los lenguajes de programación con el entorno \texttt{sourcecode}. Para usarlo, al igual que \textit{vervatim}, el objeto frame debe tener el argumento opcional \textbf{[fragile]}.

\begin{sourcecode}{python}{Ejemplo en Python.}
import numpy as np

def incmatrix(genl1, genl2):
	m = len(genl1)
	n = len(genl2)
	M = None # Comentario 1
	VT = np.zeros((n*m, 1), int) # Comentario 2
\end{sourcecode}
	
	También se pueden usar códigos inline \inlinesourcecodeboxed{javascript}{let a = (b) => \{return b*2\};} o sin recuadro de color como \inlinesourcecode{bash}{git commit -m "Este ejemplo"}.
\end{frame}

%---------------------------------------------------------------------

\begin{frame}[fragile]{Código fuente (Verbatim)}
	\begin{example}[Código sencillo en Python]
		\begin{verbatim}
		import numpy as np
		
		def incmatrix(genl1, genl2):
			m = len(genl1)
			n = len(genl2)
			M = None # Comentario 1
			VT = np.zeros((n*m, 1), int) # Comentario 2
		\end{verbatim}
	\end{example}
\end{frame}

%---------------------------------------------------------------------

\begin{frame}{Citas \& Ecuaciones}
	El template también ofrece opciones para citar \cite{template}. Análogamente, todas las fórmulas de ecuaciones de \href{https://latex.ppizarror.com/informe}{Template-Informe} son soportadas \eqref{eqn:eqn-larga}:
	
	\insertgathered[\label{eqn:eqn-larga}]{
		\lpow{\Lambda}{f} = \frac{L\cdot f}{W} \cdot \frac{\pow{\lpow{Q}{e}}{2}}{8 \pow{\pi}{2} \pow{W}{4} g} + \sum_{i=1}^{l} \frac{f \cdot \bigp{M - d}}{l \cdot W} \cdot \frac{\pow{\bigp{\lpow{Q}{e}- i\cdot Q}}{2}}{8 \pow{\pi}{2} \pow{W}{4} g}\\
		Q_e = 2.5Q \cdot \int_{0}^{e} V(x) \dd{x} + \aasin{\biggp{1+\frac{1}{1-e}}}
	}
\end{frame}

%---------------------------------------------------------------------

\begin{frame}{Teoremas}
	\begin{theorem}[Equivalencia masa energía]
		$E = mc^2$
	\end{theorem}
\end{frame}

%---------------------------------------------------------------------

\section{Segunda sección}

\begin{frame}{Contenidos}
	\tableofcontentscurrent
\end{frame}

%---------------------------------------------------------------------

% Configura el tabularframe
\tabularframecfg{0.05}{0.7}{0.4}{0.6} % Tamaño header, contenido, col. izq., col. der.
\tabularframecfgalign{false}{true} % Centrado columna izquierda, derecha
\tabularframebartextcfg{\normalsize}{true} % Tamaño, centrado

\begin{frame}{Entorno tabular (1/3)}
	\begin{tabularframehead}{1}
		\addtfheadcolumn{0.4}{Primer titular}%
		\addtfheadcolumn{0.3}{Segundo titular}%
		\addtfheadcolumn{0.3}{Tercer titular}%
	\end{tabularframehead}
	\tabularframecontent{
		\begin{itemize}[leftmargin=*]
			\item A, \cite{excel2latex}
			\item B, \cite{overleaf}
			\item C, \cite{einstein}
		\end{itemize}
	}{
		\insertimage{ejemplos/test-image.png}{scale=0.14}{Where are you? de \quotes{Internet}.}
	}
\end{frame}

\begin{frame}{Entorno tabular (2/3)}
	\tabularframeheadrepeat{2} % Repite el header anterior, pasa como argumento la columna activa
	\tabularframecontent{
		La ecuación \eqref{eqtabular} indica la profundidad de un elemento, aunque no tiene mucho sentido ya que sólo se usa dentro del template. \\
		\insertequation[\label{eqtabular}]{a = \frac{1}{2} + \cos c} 
	}{
		\lipsum[10]
	}
\end{frame}

\begin{frame}{Entorno tabular (3/3)}
	\tabularframeheadrepeat{3}
	\tabularframecontent{
		\lipsum[11]
	}{
		\begin{table}
			\centering
			\caption{Mi tabla.}
			\begin{tabular}{ccc}
				\hline
				\textbf{Columna 1} & \textbf{Columna 2} & \textbf{Columna 3} \bigstrut\\
				\hline
				$\omega$ & $\nu$ & $\delta$ \bigstrut[t]\\
				$\Phi$ & $\Theta$ & $\varSigma$ \\
				$\xi$ & $\kappa$ & $\varpi$ \bigstrut[b] \\
				\hline
			\end{tabular}
		\end{table}
	}
\end{frame}

\begin{frame}{Entorno tabular (3/3)}
	\tabularframeheadrepeat{3}
	\tabularframecfgalign{true}{true} % Configura alineación vertical (primer parámetro)
	\tabularframesinglecontent{%
		\lipsum[1]
	}
\end{frame}

%---------------------------------------------------------------------

\begin{frame}[allowframebreaks]\normalsize
	\frametitle{\namereferences}
	\bibliography{library}
\end{frame}

%---------------------------------------------------------------------

\begin{frame}
	\centering
	\Huge{Gracias por su atención}
\end{frame}