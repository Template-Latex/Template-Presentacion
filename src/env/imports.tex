% Template:     Template Presentación LaTeX
% Documento:    Importación de librerías
% Versión:      1.1.6 (24/06/2021)
% Codificación: UTF-8
%
% Autor: Pablo Pizarro R.
%        Facultad de Ciencias Físicas y Matemáticas
%        Universidad de Chile
%        pablo@ppizarror.com
%
% Sitio web:    [https://latex.ppizarror.com/presentacion]
% Licencia MIT: [https://opensource.org/licenses/MIT]

% -----------------------------------------------------------------------------
% SE GUARDAN VARIABLES ANTES DE CARGAR LIBRERÍAS
% -----------------------------------------------------------------------------
\let\RE\Re
\let\IM\Im

% -----------------------------------------------------------------------------
% PARCHES DE LIBRERÍAS
% -----------------------------------------------------------------------------
\let\counterwithout\relax
\let\counterwithin\relax
\let\underbar\relax
\let\underline\relax

% Si se desactiva el idioma
\def\unaccentedoperators {}
\def\decimalpoint {}
\def\bibname {}

\makeatletter
\def\underline#1{\relax\ifmmode\@@underline{#1}\else $\@@underline{\hbox{#1}}\m@th$\relax\fi}
\def\underbar#1{\underline{\sbox\tw@{#1}\dp\tw@\z@\box\tw@}}
\makeatother

% -----------------------------------------------------------------------------
% LIBRERÍAS DEL NÚCLEO
% -----------------------------------------------------------------------------
% Manejo de condicionales
\usepackage{ifthen}

% Carga el idioma
\ifthenelse{\equal{\usespanishbabel}{true}}{
	\usepackage[spanish,es-nosectiondot,es-lcroman,es-noquoting]{babel}}{
\ifthenelse{\equal{\useenglishbabel}{true}}{
	\usepackage[english]{babel}}{}
}

% Codificación
\ifthenelse{\equal{\compilertype}{pdf2latex}}{
	\usepackage[utf8]{inputenc}}{
}

% Lanza un mensaje de error indicando mala configuración
%	#1	Parámetros opcionales (nostop,noheader)
%	#2	Mensaje de error
% 	#3	Configuración usada
%	#4	Valores esperados
\newcommand{\throwbadconfig}[4][]{
	\ifthenelse{\equal{#1}{noheader}}{
		\errmessage{LaTeX Warning: #4}
	}{
		\ifthenelse{\equal{#1}{noheader-nostop}}{
			\errmessage{LaTeX Warning: #4}
		}{
			\errmessage{LaTeX Warning: #2 (\noexpand #3= #3). Valores esperados: #4}
		}
	}
	\ifthenelse{\equal{#1}{nostop}}{}{
		\ifthenelse{\equal{#1}{noheader-nostop}}{}{
			\stop
		}
	}
}

% Librerías matemáticas
\ifthenelse{\equal{\equationleftalign}{true}}{
	\usepackage[fleqn]{amsmath}
}{
	\usepackage{amsmath}
}

% Tamaño de la fuente del documento
\usepackage{scrextend}
\usepackage{anyfontsize}
\changefontsizes{\documentfontsize pt}

% -----------------------------------------------------------------------------
% LIBRERÍAS INDEPENDIENTES
% -----------------------------------------------------------------------------
\usepackage{amssymb}       % Librerías matemáticas
\usepackage{amsthm}        % Definición de teoremas
\usepackage{array}         % Nuevas características a las tablas
\usepackage{bigstrut}      % Líneas horizontales en tablas
\usepackage{bm}            % Caracteres en negrita en ecuaciones
\usepackage{booktabs}      % Permite manejar elementos visuales en tablas
\usepackage{caption}       % Leyendas
\usepackage{changepage}    % Condicionales para administrar páginas
\usepackage{chngcntr}      % Añade números a las leyendas
\usepackage{color}         % Colores
\usepackage{datetime}      % Fechas
\usepackage{floatpag}      % Maneja números de páginas
\usepackage{floatrow}      % Permite administrar posiciones en los caption
\usepackage{framed}        % Permite creación de recuadros
\usepackage{gensymb}       % Simbología común
\usepackage{lipsum}        % Permite crear párrafos de prueba
\usepackage{listings}      % Permite añadir código fuente
\usepackage{longtable}     % Permite utilizar tablas en varias hojas
\usepackage{mathtools}     % Permite utilizar notaciones matemáticas
\usepackage{multicol}      % Múltiples columnas
\usepackage{needspace}     % Maneja los espacios en página
\usepackage{pdflscape}     % Modo página horizontal de página
\usepackage{pdfpages}      % Permite administrar páginas en pdf
\usepackage{physics}       % Paquete de matemáticas
\usepackage{ragged2e}      % Alineación de texto
\usepackage{rotating}      % Permite rotación de objetos
\usepackage{selinput}      % Compatibilidad con acentos
\usepackage{setspace}      % Cambia el espacio entre líneas
\usepackage{soul}          % Permite subrayar texto
\usepackage{subfig}        % Permite agrupar imágenes
\usepackage{textcomp}      % Simbología común
\usepackage{url}           % Permite añadir enlaces
\usepackage{wrapfig}       % Posición de imágenes
\usepackage{xspace}        % Adminsitra espacios en párrafos y líneas

% -----------------------------------------------------------------------------
% LIBRERÍAS CON PARÁMETROS
% -----------------------------------------------------------------------------
\usepackage[makeroom]{cancel} % Cancelar términos en fórmulas
\usepackage[inline]{enumitem} % Permite enumerar ítems
\usepackage[figure,table,lstlisting]{totalcount} % Contador de objetos
\usepackage[normalem]{ulem} % Permite tachar y subrayar
\usepackage[nointegrals]{wasysym} % Contiene caracteres misceláneos (v7.0.4)
\usepackage[dvipsnames,table,usenames]{xcolor} % Paquete de colores avanzado

% -----------------------------------------------------------------------------
% LIBRERÍAS CONDICIONALES
% -----------------------------------------------------------------------------
% Imágenes en modo draft
\ifthenelse{\equal{\graphicxdraft}{true}}{
	\usepackage[
		allfiguresdraft,
		filename,
		size={scriptsize},
		style={tt}
	]{draftfigure}}{
}

% Acepta codificación UTF-8 en código fuente
\ifthenelse{\equal{\compilertype}{pdf2latex}}{
	\usepackage{listingsutf8}}{
}

% Regla superior
\ifthenelse{\equal{\footnotepagetoprule}{true}}{
	\usepackage[bottom,hang]{footmisc} % Estilo pie de página
}{
	\usepackage[bottom,norule,hang]{footmisc}
}

% Agrega punto a títulos/subtítulos
\def\showdotaftersnum {true}
\ifthenelse{\equal{\showdotaftersnum}{true}}{
	\usepackage{secdot}
	\sectiondot{subsection}
	\sectiondot{subsubsection}}{
}

% Referencias
\usepackage[pdfencoding=auto,psdextra]{hyperref}

% Anexos/Apéndices
\def\showappendixsecindex {false}
\ifthenelse{\equal{\showappendixsecindex}{true}}{
}{
	\usepackage{appendix}
}

% -----------------------------------------------------------------------------
% LIBRERÍAS DEPENDIENTES
% -----------------------------------------------------------------------------
\usepackage{bookmark}      % Administración de marcadores en pdf
\usepackage{fancyhdr}      % Encabezados y pie de páginas
\usepackage{float}         % Administrador de posiciones de objetos
\usepackage{hyperxmp}      % Etiquetas opcionales para el pdf compilado
\usepackage{multirow}      % Agrega nuevas opciones a las tablas

% -----------------------------------------------------------------------------
% TIPOGRAFÍA DEL DOCUMENTO
% -----------------------------------------------------------------------------
% Tipografías clásicas
\ifthenelse{\equal{\fontdocument}{lmodern}}{
	\usepackage{lmodern}
}{
\ifthenelse{\equal{\fontdocument}{arial}}{
	\usepackage{helvet}
	\renewcommand{\familydefault}{\sfdefault}
}{
\ifthenelse{\equal{\fontdocument}{arial2}}{
	\usepackage{arial}
}{
\ifthenelse{\equal{\fontdocument}{times}}{
	\usepackage{mathptmx}
}{
\ifthenelse{\equal{\fontdocument}{mathptmx}}{
	\usepackage{mathptmx}
}{
\ifthenelse{\equal{\fontdocument}{helvet}}{
	\renewcommand{\familydefault}{\sfdefault}
	\usepackage[scaled=0.95]{helvet}
	\usepackage[helvet]{sfmath}
}{
\ifthenelse{\equal{\fontdocument}{alegreyasans}}{
	\usepackage[sfdefault]{AlegreyaSans}
	\renewcommand*\oldstylenums[1]{{\AlegreyaSansOsF #1}}
}{
\ifthenelse{\equal{\fontdocument}{opensans}}{
	\usepackage[default,scale=0.95]{opensans}
}{
\ifthenelse{\equal{\fontdocument}{mathpazo}}{
	\usepackage{mathpazo}
}{

% Otros
\ifthenelse{\equal{\fontdocument}{accantis}}{
	\usepackage{accanthis}
}{
\ifthenelse{\equal{\fontdocument}{alegreya}}{
	\usepackage{Alegreya}
	\renewcommand*\oldstylenums[1]{{\AlegreyaOsF #1}}
}{
\ifthenelse{\equal{\fontdocument}{algolrevived}}{
	\usepackage{algolrevived}
}{
\ifthenelse{\equal{\fontdocument}{antiqua}}{
	\usepackage{antiqua}
}{
\ifthenelse{\equal{\fontdocument}{antpolt}}{
	\usepackage{antpolt}
}{
\ifthenelse{\equal{\fontdocument}{antpoltlight}}{
	\usepackage[light]{antpolt}
}{
\ifthenelse{\equal{\fontdocument}{anttor}}{
	\usepackage[math]{anttor}
}{
\ifthenelse{\equal{\fontdocument}{anttorcondensed}}{
	\usepackage[condensed,math]{anttor}
}{
\ifthenelse{\equal{\fontdocument}{anttorlight}}{
	\usepackage[light,math]{anttor}
}{
\ifthenelse{\equal{\fontdocument}{anttorlightcondensed}}{
	\usepackage[light,condensed,math]{anttor}
}{
\ifthenelse{\equal{\fontdocument}{arev}}{
	\usepackage{arev}
}{
\ifthenelse{\equal{\fontdocument}{arimo}}{
	\usepackage[sfdefault]{arimo}
	\renewcommand*\familydefault{\sfdefault}
}{
\ifthenelse{\equal{\fontdocument}{aurical}}{
	\usepackage{aurical}
}{
\ifthenelse{\equal{\fontdocument}{avant}}{
	\usepackage{avant}
}{
\ifthenelse{\equal{\fontdocument}{baskervald}}{
	\usepackage{baskervald}
}{
\ifthenelse{\equal{\fontdocument}{berasans}}{
	\usepackage[scaled]{berasans}
	\renewcommand*\familydefault{\sfdefault}
}{
\ifthenelse{\equal{\fontdocument}{beraserif}}{
	\usepackage{bera}
}{
\ifthenelse{\equal{\fontdocument}{biolinum}}{
	\usepackage{libertine}
	\renewcommand*\familydefault{\sfdefault}
}{
\ifthenelse{\equal{\fontdocument}{cabin}}{
	\usepackage[sfdefault]{cabin}
	\renewcommand*\familydefault{\sfdefault}
}{
\ifthenelse{\equal{\fontdocument}{cabincondensed}}{
	\usepackage[sfdefault,condensed]{cabin}
	\renewcommand*\familydefault{\sfdefault}
}{
\ifthenelse{\equal{\fontdocument}{cantarell}}{
	\usepackage[default]{cantarell}
}{
\ifthenelse{\equal{\fontdocument}{caladea}}{
	\usepackage{caladea}
}{
\ifthenelse{\equal{\fontdocument}{carlito}}{
	\usepackage[sfdefault]{carlito}
	\renewcommand*\familydefault{\sfdefault}
}{
\ifthenelse{\equal{\fontdocument}{chivolight}}{
	\usepackage[familydefault,light]{Chivo}
}{
\ifthenelse{\equal{\fontdocument}{chivoregular}}{
	\usepackage[familydefault,regular]{Chivo}
}{
\ifthenelse{\equal{\fontdocument}{clearsans}}{
	\usepackage[sfdefault]{ClearSans}
	\renewcommand*\familydefault{\sfdefault}
}{
\ifthenelse{\equal{\fontdocument}{comfortaa}}{
	\usepackage[default]{comfortaa}
}{
\ifthenelse{\equal{\fontdocument}{comicneue}}{
	\usepackage[default]{comicneue}
}{
\ifthenelse{\equal{\fontdocument}{comicneueangular}}{
	\usepackage[default,angular]{comicneue}
}{
\ifthenelse{\equal{\fontdocument}{crimson}}{
	\usepackage{crimson}
}{
\ifthenelse{\equal{\fontdocument}{cyklop}}{
	\usepackage{cyklop}
}{
\ifthenelse{\equal{\fontdocument}{dejavusans}}{
	\usepackage{DejaVuSans}
	\renewcommand*\familydefault{\sfdefault}
}{
\ifthenelse{\equal{\fontdocument}{dejavusanscondensed}}{
	\usepackage{DejaVuSansCondensed}
	\renewcommand*\familydefault{\sfdefault}
}{
\ifthenelse{\equal{\fontdocument}{droidsans}}{
	\usepackage[defaultsans]{droidsans}
	\renewcommand*\familydefault{\sfdefault}
}{
\ifthenelse{\equal{\fontdocument}{fetamont}}{
	\usepackage{fetamont}
	\renewcommand*\familydefault{\sfdefault}
}{
\ifthenelse{\equal{\fontdocument}{firasans}}{
	\usepackage[sfdefault]{FiraSans}
	\renewcommand*\familydefault{\sfdefault}
}{
\ifthenelse{\equal{\fontdocument}{iwona}}{
	\usepackage[math]{iwona}
}{
\ifthenelse{\equal{\fontdocument}{iwonacondensed}}{
	\usepackage[math]{iwona}
}{
\ifthenelse{\equal{\fontdocument}{iwonalight}}{
	\usepackage[light,math]{iwona}
}{
\ifthenelse{\equal{\fontdocument}{iwonalightcondensed}}{
	\usepackage[light,condensed,math]{iwona}
}{
\ifthenelse{\equal{\fontdocument}{kurier}}{
	\usepackage[math]{kurier}
}{
\ifthenelse{\equal{\fontdocument}{kuriercondensed}}{
	\usepackage[condensed,math]{kurier}
}{
\ifthenelse{\equal{\fontdocument}{kurierlight}}{
	\usepackage[light,math]{kurier}
}{
\ifthenelse{\equal{\fontdocument}{kurierlightcondensed}}{
	\usepackage[light,condensed,math]{kurier}
}{
\ifthenelse{\equal{\fontdocument}{lato}}{
	\usepackage[default]{lato}
}{
\ifthenelse{\equal{\fontdocument}{libris}}{
	\usepackage{libris}
	\renewcommand*\familydefault{\sfdefault}
}{
\ifthenelse{\equal{\fontdocument}{lxfonts}}{
	\usepackage{lxfonts}
}{
\ifthenelse{\equal{\fontdocument}{merriweather}}{
	\usepackage[sfdefault]{merriweather}
}{
\ifthenelse{\equal{\fontdocument}{merriweatherlight}}{
	\usepackage[sfdefault,light]{merriweather}
}{
\ifthenelse{\equal{\fontdocument}{mintspirit}}{
	\usepackage[default]{mintspirit}
}{
\ifthenelse{\equal{\fontdocument}{montserratalternatesextralight}}{
	\usepackage[defaultfam,extralight,tabular,lining,alternates]{montserrat}
	\renewcommand*\oldstylenums[1]{{\fontfamily{Montserrat-TOsF}\selectfont #1}}
}{
\ifthenelse{\equal{\fontdocument}{montserratalternatesregular}}{
	\usepackage[defaultfam,tabular,lining,alternates]{montserrat}
	\renewcommand*\oldstylenums[1]{{\fontfamily{Montserrat-TOsF}\selectfont #1}}
}{
\ifthenelse{\equal{\fontdocument}{montserratalternatesthin}}{
	\usepackage[defaultfam,thin,tabular,lining,alternates]{montserrat}
	\renewcommand*\oldstylenums[1]{{\fontfamily{Montserrat-TOsF}\selectfont #1}}
}{
\ifthenelse{\equal{\fontdocument}{montserratextralight}}{
	\usepackage[defaultfam,extralight,tabular,lining]{montserrat}
	\renewcommand*\oldstylenums[1]{{\fontfamily{Montserrat-TOsF}\selectfont #1}}
}{
\ifthenelse{\equal{\fontdocument}{montserratlight}}{
	\usepackage[defaultfam,light,tabular,lining]{montserrat}
	\renewcommand*\oldstylenums[1]{{\fontfamily{Montserrat-TOsF}\selectfont #1}}
}{
\ifthenelse{\equal{\fontdocument}{montserratregular}}{
	\usepackage[defaultfam,tabular,lining]{montserrat}
	\renewcommand*\oldstylenums[1]{{\fontfamily{Montserrat-TOsF}\selectfont #1}}
}{
\ifthenelse{\equal{\fontdocument}{montserratthin}}{
	\usepackage[defaultfam,thin,tabular,lining]{montserrat}
	\renewcommand*\oldstylenums[1]{{\fontfamily{Montserrat-TOsF}\selectfont #1}}
}{
\ifthenelse{\equal{\fontdocument}{nimbussans}}{
	\usepackage{nimbussans}
	\renewcommand*\familydefault{\sfdefault}
}{
\ifthenelse{\equal{\fontdocument}{noto}}{
	\usepackage[sfdefault]{noto}
	\renewcommand*\familydefault{\sfdefault}
}{
\ifthenelse{\equal{\fontdocument}{opensansserif}}{
	\usepackage[default,oldstyle,scale=0.95]{opensans}
}{
\ifthenelse{\equal{\fontdocument}{overlock}}{
	\usepackage[sfdefault]{overlock}
	\renewcommand*\familydefault{\sfdefault}
}{
\ifthenelse{\equal{\fontdocument}{paratype}}{
	\usepackage{paratype}
	\renewcommand*\familydefault{\sfdefault}
}{
\ifthenelse{\equal{\fontdocument}{paratypesanscaption}}{
	\usepackage{PTSansCaption}
	\renewcommand*\familydefault{\sfdefault}
}{
\ifthenelse{\equal{\fontdocument}{paratypesansnarrow}}{
	\usepackage{PTSansNarrow}
	\renewcommand*\familydefault{\sfdefault}
}{
\ifthenelse{\equal{\fontdocument}{quattrocento}}{
	\usepackage[sfdefault]{quattrocento}
}{
\ifthenelse{\equal{\fontdocument}{raleway}}{
	\usepackage[default]{raleway}
}{
\ifthenelse{\equal{\fontdocument}{roboto}}{
	\usepackage[sfdefault]{roboto}
}{
\ifthenelse{\equal{\fontdocument}{robotocondensed}}{
	\usepackage[sfdefault,condensed]{roboto}
}{
\ifthenelse{\equal{\fontdocument}{robotolight}}{
	\usepackage[sfdefault,light]{roboto}
}{
\ifthenelse{\equal{\fontdocument}{robotolightcondensed}}{
	\usepackage[sfdefault,light,condensed]{roboto}
}{
\ifthenelse{\equal{\fontdocument}{robotothin}}{
	\usepackage[sfdefault,thin]{roboto}
}{
\ifthenelse{\equal{\fontdocument}{rosario}}{
	\usepackage[familydefault]{Rosario}
}{
\ifthenelse{\equal{\fontdocument}{sourcesanspro}}{
	\usepackage[default]{sourcesanspro}
}{
\ifthenelse{\equal{\fontdocument}{uarial}}{
	\usepackage{uarial}
	\renewcommand*\familydefault{\sfdefault}
}{
\ifthenelse{\equal{\fontdocument}{ugq}}{
	\renewcommand*\sfdefault{ugq}
	\renewcommand*\familydefault{\sfdefault}
}{
\ifthenelse{\equal{\fontdocument}{universalis}}{
	\usepackage[sfdefault]{universalis}
}{
\ifthenelse{\equal{\fontdocument}{universaliscondensed}}{
	\usepackage[condensed,sfdefault]{universalis}
}{
\ifthenelse{\equal{\fontdocument}{venturis}}{
	\usepackage[lf]{venturis}
	\renewcommand*\familydefault{\sfdefault}
}{
	\throwbadconfig[nostop]{Fuente desconocida}{\fontdocument}{(Fuentes recomendadas) lmodern,arial,arial2,helvet,times,alegreyasans,mathpazo,mathptmx}
	\throwbadconfig[noheader-nostop]{Fuente desconocida}{\fontdocument}{(Fuentes adicionales) accantis,alegreya,algolrevived,antiqua,antpolt,antpoltlight,anttor,anttorcondensed,anttorlight,anttorlightcondensed,arev,arimo,aurical,avant,baskervald,berasans,beraserif,biolinum,cabin,cabincondensed,cantarell,caladea,carlito,chivolight,chivoregular,clearsans,comfortaa,comicneue,comicneueangular,crimson,cyklop,dejavusans,dejavusanscondensed,droidsans,firasans,iwona,iwonacondensed,iwonalight,iwonalightcondensed,kurier}
	\throwbadconfig[noheader-nostop]{Fuente desconocida}{\fontdocument}{kuriercondensed,kurierlight,kurierlightcondensed,lato,libris,lxfonts,merriweather,merriweatherlight,mintspirit,montserratalternatesextralight,montserratalternatesregular,montserratalternatesthin,montserratextralight,montserratlight,montserratregular,montserratthin,nimbussans,noto,opensansserif,overlock,paratype,paratypesanscaption,paratypesansnarrow,quattrocento,raleway,roboto,robotolight,robotolightcondensed,robotothin,rosario,sourcesanspro,uarial,ugq}
	\throwbadconfig[noheader]{Fuente desconocida}{\fontdocument}{universalis,universaliscondensed,venturis}
	}}}}}}}}}}}}}}}}}}}}}}}}}}}}}}}}}}}}}}}}}}}}}}}}}}}}}}}}}}}}}}}}}}}}}}}}}}}}}}}}}}}}}}
}

% -----------------------------------------------------------------------------
% TIPOGRAFÍA TYPEWRITER
% -----------------------------------------------------------------------------
\ifthenelse{\equal{\fonttypewriter}{tmodern}}{
	\renewcommand*\ttdefault{lmvtt}
}{
\ifthenelse{\equal{\fonttypewriter}{anonymouspro}}{
	\usepackage[ttdefault=true]{AnonymousPro}
}{
\ifthenelse{\equal{\fonttypewriter}{ascii}}{
	\usepackage{ascii}
	\let\SI\relax
}{
\ifthenelse{\equal{\fonttypewriter}{beramono}}{
	\usepackage[scaled]{beramono}
}{
\ifthenelse{\equal{\fonttypewriter}{cmpica}}{
	\usepackage{addfont}
	\addfont{OT1}{cmpica}{\pica}
	\addfont{OT1}{cmpicab}{\picab}
	\addfont{OT1}{cmpicati}{\picati}
	\renewcommand*\ttdefault{pica}
}{
\ifthenelse{\equal{\fonttypewriter}{courier}}{
	\usepackage{courier}
}{
\ifthenelse{\equal{\fonttypewriter}{dejavusansmono}}{
	\usepackage[scaled]{DejaVuSansMono}
}{
\ifthenelse{\equal{\fonttypewriter}{firamono}}{
	\usepackage[scale=0.85]{FiraMono}
}{
\ifthenelse{\equal{\fonttypewriter}{gomono}}{
	\usepackage[scale=0.85]{GoMono}
}{
\ifthenelse{\equal{\fonttypewriter}{inconsolata}}{
	\usepackage{inconsolata}
}{
\ifthenelse{\equal{\fonttypewriter}{nimbusmono}}{
	\usepackage{nimbusmono}
}{
\ifthenelse{\equal{\fonttypewriter}{newtxtt}}{
	\usepackage[zerostyle=d]{newtxtt}
}{
\ifthenelse{\equal{\fonttypewriter}{nimbusmono}}{
	\usepackage{nimbusmono}
}{
\ifthenelse{\equal{\fonttypewriter}{nimbusmononarrow}}{
	\usepackage{nimbusmononarrow}
}{
\ifthenelse{\equal{\fonttypewriter}{lcmtt}}{
	\renewcommand*\ttdefault{lcmtt}
}{
\ifthenelse{\equal{\fonttypewriter}{sourcecodepro}}{
	\usepackage[ttdefault=true,scale=0.85]{sourcecodepro}
}{
\ifthenelse{\equal{\fonttypewriter}{texgyrecursor}}{
	\usepackage{tgcursor}
}{
	\throwbadconfig{Fuente desconocida}{\fonttypewriter}{anonymouspro,ascii,beramono,
		cmpica,courier,dejavusansmono,firamono,gomono,inconsolata,kpmonospaced,lcmtt,
		newtxtt,nimbusmono,nimbusmononarrow,texgyrecursor,tmodern}
	}}}}}}}}}}}}}}}}
}

% -----------------------------------------------------------------------------
% FINALES
% -----------------------------------------------------------------------------
\usepackage[T1]{fontenc} % Caracteres acentuados
\ifthenelse{\equal{\showlayoutlines}{true}}{ % Muestra las líneas del layout
	\usepackage{showframe}}{
}
\def\showlinenumbers {true}
\ifthenelse{\equal{\showlinenumbers}{true}}{ % Muestra los números de línea
	\usepackage[switch,columnwise,running]{lineno}}{
}
\usepackage{csquotes} % Citas y comillas, se debe usar después de lineno [6.4.2]
\ifthenelse{\equal{\compilertype}{pdf2latex}}{
	\inputencoding{utf8}}{
}

\usefonttheme{professionalfonts}
\usepackage{transparent}

