% Template:     Presentación LaTeX
% Documento:    Funciones del template
% Versión:      2.2.0 (29/05/2023)
% Codificación: UTF-8
%
% Autor: Pablo Pizarro R.
%        pablo@ppizarror.com
%
% Manual template: [https://latex.ppizarror.com/presentacion]
% Licencia MIT:    [https://opensource.org/licenses/MIT]

% Importación de librerías
\usepackage{changepage}

% Crea variables para guardar configuraciones del tabularframe
\global\def\GLOBALtabularframebarcreated {false}
\global\def\GLOBALtabularframebarfontsize {\normalsize}
\global\def\GLOBALtabularframebarheight {0.05}
\global\def\GLOBALtabularframebarheightmargin {0}
\global\def\GLOBALtabularframebartextcentered {true}
\global\def\GLOBALtabularframecontentheight {0.65}
\global\def\GLOBALtabularframecontentleftcentered {false}
\global\def\GLOBALtabularframecontentleftfontsize {\normalsize}
\global\def\GLOBALtabularframecontentrightcentered {false}
\global\def\GLOBALtabularframecontentrightfontsize {\normalsize}
\global\def\GLOBALtabularframecontentwidthleft {0.4}
\global\def\GLOBALtabularframecontentwidthright {0.6}

\def\LOCALtabularframecolumnselected {1}
\def\LOCALtabularframeinit {false}
\def\LOCALtabularframestorevar {true}

% Archivo que guarda el código generado del framehead
\newwrite\filetabularframehead

% Define contadores
\newcounter{tabularcolnumber}
\setcounter{tabularcolnumber}{0}

% Define tamaños
\newlength{\coretabularheightsize}

% Función que configura el texto de la barra superior del tabular frame
% 	#1 Tamaño de letra
%	#2 Texto centrado
\newcommand{\tabularframebartextcfg}[2]{%
	\emptyvarerr{\tabularframebartextcfg}{#1}{Tamano fuente no definida}%
	\emptyvarerr{\tabularframebartextcfg}{#2}{Texto centrado no definido}%
	\corecheckbooleanvar{#2}%
	\global\def\GLOBALtabularframebarfontsize {#1}%
	\global\def\GLOBALtabularframebartextcentered {#2}%
}

% Función que configura el entorno tabular frames
%	#1	Margen superior del head en porcentaje del alto de la página
%	#2	Altura de head en porcentaje (0-1)
%	#3	Altura de content en porcentaje (0-1)
%	#4	Ancho de la barra izquierda de contenidos (coloreada)
%	#5	Ancho de la barra derecha de contenidos
\newcommand{\tabularframecfg}[5][]{%
	\emptyvarerr{\tabularframecfg}{#2}{Altura de head no definido}%
	\emptyvarerr{\tabularframecfg}{#3}{Altura de content no definido}%
	\emptyvarerr{\tabularframecfg}{#4}{Ancho de la barra izquierda de contenidos}%
	\emptyvarerr{\tabularframecfg}{#5}{Ancho de la barra derecha de contenidos}%
	\ifthenelse{\equal{#1}{}}{}{%
		\global\def\GLOBALtabularframebarheightmargin {#1}%
	}%
	\global\def\GLOBALtabularframebarheight {#2}%
	\global\def\GLOBALtabularframecontentheight {#3}%
	\global\def\GLOBALtabularframecontentwidthleft {#4}%
	\global\def\GLOBALtabularframecontentwidthright {#5}%
	\setlength{\coretabularheightsize}{\GLOBALtabularframebarheight\textheight}%
}

% Función que configura la alineación del entorno tabular frames
%	#1	Columna izquierda contenidos centrada (true/false)
%	#2	Columna derecha contenidos centrada (true/false)
\newcommand{\tabularframecfgalign}[2]{%
	\emptyvarerr{\tabularframecfgalign}{#1}{Centrado columna izquierda no definido}%
	\emptyvarerr{\tabularframecfgalign}{#2}{Centrado columna derecha no definido}%
	\corecheckbooleanvar{#1}%
	\corecheckbooleanvar{#2}%
	\global\def\GLOBALtabularframecontentleftcentered {#1}%
	\global\def\GLOBALtabularframecontentrightcentered {#2}%
}

% Función que configura el tamaño de fuente de los tabular frames
%	#1	Tamaño fuente columna izquierda
%	#2	Tamaño fuente columna derecha
\newcommand{\tabularframecfgfontsize}[2]{%
	\emptyvarerr{\tabularframecfgfontsize}{#1}{Tamano fuente columna izquierda no definido}%
	\emptyvarerr{\tabularframecfgfontsize}{#2}{Tamano fuente columna derecha no definido}%
	\global\def\GLOBALtabularframecontentleftfontsize {#1}%
	\global\def\GLOBALtabularframecontentrightfontsize {#2}%
}

% Crea el header del tabular frame, acepta columnas creadas con \addtfheadcolumn
%	#1	Número de la columna destacada (desde 1)
\newenvironment{tabularframehead}[1]{%
	\emptyvarerr{\tabularframehead}{#1}{Numero de columna destacada no definido}%
	\def\LOCALtabularframecolumnselected {#1}%
	\def\LOCALtabularframeinit {true}%
	\setcounter{tabularcolnumber}{0}%
	\ifthenelse{\equal{\LOCALtabularframestorevar}{true}}{%
		\immediate\openout\filetabularframehead=\jobname.tframeh%
	}{}%
	\vspace{\dimexpr-0.5\coretabularheightsize + \GLOBALtabularframebarheightmargin\textheight}%
	\begin{columns}[t,onlytextwidth]%
	}{%
	\end{columns}
	\setcounter{tabularcolnumber}{-1}%
	\ifthenelse{\equal{\LOCALtabularframestorevar}{true}}{%
		\immediate\closeout\filetabularframehead%
		\global\def\GLOBALtabularframebarcreated {true}%
	}{}%
}

% Función que repite el último entorno tabular frames
%	#1	Número de la columna destacada (desde 1)
\newcommand{\tabularframeheadrepeat}[1]{
	\ifthenelse{\equal{\GLOBALtabularframebarcreated}{true}}{}{%
		\throwwarning{Funcion \noexpand\tabularframeheadrepeat no puede usarse si no existe un \noexpand\tabularframehead previo}\stop}%
	\emptyvarerr{\tabularframeheadrepeat}{#1}{Numero de columna destacada no definido}%
	\def\LOCALtabularframestorevar {false}%
	\begin{tabularframehead}{#1}%
		\input{\jobname.tframeh}%
	\end{tabularframehead}
	\def\LOCALtabularframestorevar {true}%
}

% Agrega una columna al tabular head
%	#1	Tamaño de la columna
%	#2	Texto
\newcommand{\addtfheadcolumn}[2]{%
	\ifthenelse{\equal{\LOCALtabularframeinit}{true}}{}{%
		\throwwarning{Funcion \noexpand\addtfheadcolumn no puede usarse fuera del entorno \noexpand\tabularframehead}\stop}%
	\emptyvarerr{\addtfheadcolumn}{#1}{Tamano de la columna no definido}%
	\emptyvarerr{\addtfheadcolumn}{#2}{Texto de la columna no definido}%
	\ifthenelse{\equal{\LOCALtabularframestorevar}{true}}{%
		\immediate\write\filetabularframehead{\unexpanded{\addtfheadcolumn{#1}{#2}}\LOCALpercentchar}%
	}{}%
	\stepcounter{tabularcolnumber}%
	\column{#1\linewidth}%
	\ifthenelse{\equal{\thetabularcolnumber}{\LOCALtabularframecolumnselected}}{%
		\transparent{1.0}%
	}{%
		\transparent{\tabularframeopacity}%
	}%
	\usebeamercolor[fg]{frametitle}\colorbox{bg}{\begin{minipage}[t][\GLOBALtabularframebarheight\textheight][t]{\dimexpr\linewidth - 15\fboxrule}%
		\usebeamercolor[fg]{frametitle}{%
			\ifthenelse{\equal{\GLOBALtabularframebartextcentered}{false}}{%
				\verticallycentertext{\GLOBALtabularframebarfontsize{#2}}%
			}{%
				\begin{center}%
					\verticallycentertext{\GLOBALtabularframebarfontsize{#2}}%
				\end{center}
			}%
		}%
	\end{minipage}}%
}

% Inicia el contenido de un tabular frame
%	#1	Panel izquierdo
%	#2	Panel derecho
\newcommand{\tabularframecontent}[2]{%
	\ifthenelse{\equal{\thetabularcolnumber}{-1}}{}{%
		\throwwarning{Funcion \noexpand\tabularframecontent no puede usarse sin haber inicializado \noexpand\tabularframehead}\stop}%
	\emptyvarerr{\tabularframecontent}{#1}{Columna izquierda no definida}%
	\emptyvarerr{\tabularframecontent}{#2}{Columna derecha no definida}%
	\vspace{-2.5\fboxrule}%
	\begin{columns}[t,onlytextwidth]%
		\column{\GLOBALtabularframecontentwidthleft\linewidth}%
		\usebeamercolor[fg]{block body}\colorbox{bg}%
		{\begin{minipage}[t][\GLOBALtabularframecontentheight \textheight][t]{\dimexpr\textwidth - 15\fboxrule}%
			\vspace{\tabularframepaddingtop em}%
			\hspace{\tabularframepaddingleft em}%
			\begin{minipage}[t][\dimexpr\GLOBALtabularframecontentheight \textheight-\tabularframepaddingtop em][t]{\dimexpr\linewidth - \tabularframepaddingleft em - \tabularframepaddingright em}
				\GLOBALtabularframecontentleftfontsize%
				\ifthenelse{\equal{\GLOBALtabularframecontentleftcentered}{false}}{%
					#1%
				}{%
					\verticallycentertext{#1}%
				}%
			\end{minipage}
		\end{minipage}}
		\column{\GLOBALtabularframecontentwidthright\linewidth}%
		\centering
		{\begin{minipage}[t][\GLOBALtabularframecontentheight \textheight][t]{\dimexpr\textwidth - 10\fboxrule}%
			\vspace{\tabularframepaddingtop em}%
			\hspace{\tabularframepaddingleft em}%
			\begin{minipage}[t][\dimexpr\GLOBALtabularframecontentheight \textheight-\tabularframepaddingtop em][t]{\dimexpr\linewidth - \tabularframepaddingleft em - \tabularframepaddingright em}
				\GLOBALtabularframecontentrightfontsize%
				\ifthenelse{\equal{\GLOBALtabularframecontentrightcentered}{false}}{%
					#2%
				}{%
					\vspace{-0.5\baselineskip}%
					\verticallycentertext{#2}%
				}%
			\end{minipage}
		\end{minipage}}
	\end{columns}
	\def\LOCALtabularframeinit {false}%
}

% Tabular frame de una sola columna
%	#1	Color, si vacío usa por defecto el color de columna izquierda
%	#2	Contenido
\newcommand{\tabularframesinglecontent}[2][]{%
	\ifthenelse{\equal{\thetabularcolnumber}{-1}}{}{%
		\throwwarning{Funcion \noexpand\tabularframesinglecontent no puede usarse sin haber inicializado \noexpand\tabularframehead}\stop}%
	\emptyvarerr{\tabularframesinglecontent}{#2}{Contenido no definido}%
	\vspace{-2.5\fboxrule}%
	\ifx\hfuzz#1\hfuzz%
		\def\tabularframesinglecontentemptycolor {bg}%
		\usebeamercolor[fg]{block body}%
	\else%
		\def\tabularframesinglecontentemptycolor {#1}%
	\fi%
	\begin{columns}[t,onlytextwidth]%
		\column{1\linewidth}%
		\colorbox{\tabularframesinglecontentemptycolor}%
		{\begin{minipage}[t][\GLOBALtabularframecontentheight \textheight][t]{\dimexpr\textwidth - 15\fboxrule}%
			\vspace{\tabularframepaddingtop em}%
			\hspace{\tabularframepaddingleft em}%
			\begin{minipage}[t][\dimexpr\GLOBALtabularframecontentheight \textheight-\tabularframepaddingtop em][t]{\dimexpr\linewidth - \tabularframepaddingleft em - \tabularframepaddingright em}%
				\GLOBALtabularframecontentleftfontsize%
				\ifthenelse{\equal{\GLOBALtabularframecontentleftcentered}{false}}{%
					#2%
				}{%
					\vspace{-0.5\baselineskip}%
					\verticallycentertext{#2}%
				}%
			\end{minipage}
		\end{minipage}}
	\end{columns}
	\def\LOCALtabularframeinit {false}%
}
