% Template:     Template Presentación LaTeX
% Documento:    Configuración de página
% Versión:      1.0.0 (17/05/2021)
% Codificación: UTF-8
%
% Autor: Pablo Pizarro R.
%        Facultad de Ciencias Físicas y Matemáticas
%        Universidad de Chile
%        pablo@ppizarror.com
%
% Sitio web:    [https://latex.ppizarror.com/presentacion]
% Licencia MIT: [https://opensource.org/licenses/MIT]

\newcommand{\templatePagecfg}{
    \setcounter{page}{1}
    \setcounter{footnote}{0}
\def\arraystretch {\tablepaddingv}
\setlength{\tabcolsep}{\tablepaddingh em}
	\ifthenelse{\equal{\pointdecimal}{true}}{
		\decimalpoint}{
	}
\renewcommand{\appendixname}{\nomltappendixsection}
\renewcommand{\figurename}{\nomltwfigure}
\renewcommand{\lstlistingname}{\nomltwsrc}
\renewcommand{\refname}{\namereferences}
\renewcommand{\bibname}{\namereferences}
\renewcommand{\tablename}{\nomltwtable}
\def\bibfont{\fontsizerefbibl}
	\ifthenelse{\equal{\stylecitereferences}{apacite}}{
		\renewcommand{\BOthers}[1]{\apacitebothers\hbox{}}
	}{}
	\ifthenelse{\equal{\showlinenumbers}{true}}{
		\linenumbers}{
	}
	\ifthenelse{\isundefined{\abstractname}}{
		\newcommand{\abstractname}{\nameabstract}
		\throwwarning{La variable \noexpand\abstractname no existe, lo que indica que la libreria babel no se ha cargado. Si ha desactivado la configuracion \noexpand\usespanishbabel debe cargar manualmente la libreria babel con algun otro idioma, como por ejemplo \noexpand\usepackage[english]{babel}, o bien define en true la configuracion \noexpand\useenglishbabel}
	}{
		\renewcommand{\abstractname}{\nameabstract}
	}
}
