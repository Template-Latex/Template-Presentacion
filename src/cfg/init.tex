% Template:     Presentación LaTeX
% Documento:    Configuración inicial del template
% Versión:      2.1.5 (06/03/2023)
% Codificación: UTF-8
%
% Autor: Pablo Pizarro R.
%        pablo@ppizarror.com
%
% Manual template: [https://latex.ppizarror.com/presentacion]
% Licencia MIT:    [https://opensource.org/licenses/MIT]

% Se revisa si las variables no han sido borradas
\checkvardefined{\coursecode}
\checkvardefined{\coursename}
\checkvardefined{\documentauthor}
\checkvardefined{\documentsubject}
\checkvardefined{\universitydepartment}
\checkvardefined{\universityfaculty}
\checkvardefined{\universitylocation}
\checkvardefined{\universityname}

% -----------------------------------------------------------------------------
% Se añade \xspace a las variables
% -----------------------------------------------------------------------------
\makeatletter
	\g@addto@macro\coursecode\xspace
	\g@addto@macro\coursename\xspace
	\g@addto@macro\documentauthor\xspace
	\g@addto@macro\documentsubject\xspace
	\g@addto@macro\universitydepartment\xspace
	\g@addto@macro\universityfaculty\xspace
	\g@addto@macro\universitylocation\xspace
	\g@addto@macro\universityname\xspace
\makeatother

% -----------------------------------------------------------------------------
% Se activan números en menú marcadores del pdf
% -----------------------------------------------------------------------------
\ifthenelse{\equal{\cfgpdfsecnumbookmarks}{true}}{
	\bookmarksetup{numbered}}{
}

% -----------------------------------------------------------------------------
% Se define metadata del pdf
% -----------------------------------------------------------------------------
\ifthenelse{\equal{\cfgshowbookmarkmenu}{true}}{
	\def\cfgpdfpagemode {UseOutlines}
	}{
	\def\cfgpdfpagemode {UseNone}
}
\ifthenelse{\equal{\usepdfmetadata}{true}}{
	\def\pdfmetainfoauthor {\documentauthor}
	\def\pdfmetainfocoursecode {\coursecode}
	\def\pdfmetainfocoursename {\coursename}
	\def\pdfmetainfosubject {\documentsubject}
	\def\pdfmetainfotitle {\documenttitle}
	\def\pdfmetainfouniversity {\universityname}
	\def\pdfmetainfouniversitydepartment {\universitydepartment}
	\def\pdfmetainfouniversityfaculty {\universityfaculty}
	\def\pdfmetainfouniversitylocation {\universitylocation}
}{
	\def\pdfmetainfoauthor {}
	\def\pdfmetainfocoursecode {}
	\def\pdfmetainfocoursename {}
	\def\pdfmetainfosubject {}
	\def\pdfmetainfotitle {}
	\def\pdfmetainfouniversity {}
	\def\pdfmetainfouniversitydepartment {}
	\def\pdfmetainfouniversityfaculty {}
	\def\pdfmetainfouniversitylocation {}
}
\hypersetup{
	keeppdfinfo,
	bookmarksopen={\cfgpdfbookmarkopen},
	bookmarksopenlevel={\cfgbookmarksopenlevel},
	bookmarkstype={toc},
	pdfauthor={\pdfmetainfoauthor},
	pdfcenterwindow={\cfgpdfcenterwindow},
	pdfcopyright={\cfgpdfcopyright},
	pdfcreator={LaTeX},
	pdfdisplaydoctitle={\cfgpdfdisplaydoctitle},
	pdfencoding={unicode},
	pdffitwindow={\cfgpdffitwindow},
	pdfinfo={
		Course.Code={\pdfmetainfocoursecode},
		Course.Name={\pdfmetainfocoursename},
		Document.Author={\pdfmetainfoauthor},
		Document.Subject={\pdfmetainfosubject},
		Document.Title={\pdfmetainfotitle},
		Template.Author.Alias={ppizarror},
		Template.Author.Email={pablo@ppizarror.com},
		Template.Author.Web={https://ppizarror.com},
		Template.Author={Pablo Pizarro R.},
		Template.Date={06/03/2023},
		Template.Encoding={UTF-8},
		Template.Latex.Compiler={pdflatex},
		Template.License.Type={MIT},
		Template.License.Web={https://opensource.org/licenses/MIT},
		Template.Name={Template-Presentacion},
		Template.Type={Normal},
		Template.Version.Dev={2.1.5-PRES},
		Template.Version.Hash={16951026740CD6DB9DAE565130345E69},
		Template.Version.Release={2.1.5},
		Template.Web.Dev={https://github.com/Template-Latex/Template-Presentacion},
		Template.Web.Manual={https://latex.ppizarror.com/presentacion},
		University.Department={\pdfmetainfouniversitydepartment},
		University.Faculty={\pdfmetainfouniversityfaculty},
		University.Location={\pdfmetainfouniversitylocation},
		University.Name={\pdfmetainfouniversity}
	},
	pdfkeywords={\cfgpdfkeywords},
	pdfmenubar={\cfgpdfmenubar},
	pdfpagelayout={\cfgpdflayout},
	pdfpagemode={\cfgpdfpagemode},
	pdfproducer={Template-Presentacion v2.1.5 | (Pablo Pizarro R.) ppizarror.com},
	pdfremotestartview={Fit},
	pdfstartpage={1},
	pdfstartview={\cfgpdfpageview},
	pdfsubject={\pdfmetainfosubject},
	pdftitle={\pdfmetainfotitle},
	pdftoolbar={\cfgpdftoolbar}
}

% -----------------------------------------------------------------------------
% Establece la carpeta de imágenes por defecto
% -----------------------------------------------------------------------------
\graphicspath{{./\defaultimagefolder}}

% -----------------------------------------------------------------------------
% Elimina el espacio vertical de los flotantes
% -----------------------------------------------------------------------------
\makeatletter
\ifthenelse{\equal{\fpremovetopbottomcenter}{true}}{
	\setlength{\@fptop}{0pt}
	\setlength{\@fpbot}{0pt}
}{}
\makeatother

% -----------------------------------------------------------------------------
% Definición de valores e dimensiones
% -----------------------------------------------------------------------------
\setstretch{\documentinterline} % Ajuste del entrelineado
\setlength{\headheight}{64 pt} % Tamaño de la cabecera sin fancyhdr
\setlength{\columnsep}{\columnsepwidth em} % Separación entre columnas
\ifthenelse{\equal{\showlinenumbers}{true}}{
	\setlength{\linenumbersep}{\marginlinenumbers pt}
	\renewcommand\linenumberfont{\normalfont\tiny\color{\linenumbercolor}}
	}{
}

% -----------------------------------------------------------------------------
% Posición inicial de los objetos
% -----------------------------------------------------------------------------
\floatplacement{figure}{\imagedefaultplacement}
\floatplacement{table}{\tabledefaultplacement}
\floatplacement{tikz}{\tikzdefaultplacement}

% -----------------------------------------------------------------------------
% Configuración de los colores
% -----------------------------------------------------------------------------
\color{\maintextcolor} % Color principal
\arrayrulecolor{\tablelinecolor} % Color de las líneas de las tablas
\sethlcolor{\highlightcolor} % Color del subrayado por defecto
\ifthenelse{\equal{\showborderonlinks}{true}}{
	% Color de links con borde
	\hypersetup{
		citebordercolor=\numcitecolor,
		linkbordercolor=\linkcolor,
		urlbordercolor=\urlcolor
	}
}{
	% Color de links sin borde
	\hypersetup{ % No reorganizar
		hidelinks,
		colorlinks=true,
		citecolor=\numcitecolor,
		filecolor=\urlcolor,
		linkcolor=\linkcolor,
		urlcolor=\urlcolor
	}
}

% -----------------------------------------------------------------------------
% Configuración de las leyendas
% -----------------------------------------------------------------------------
% Márgenes de las leyendas por defecto
\setcaptionmargincm{\captionlrmargin}
\ifthenelse{\equal{\captiontextbold}{true}}{ % Texto en negrita en etiquetas
	\renewcommand{\captiontextbold}{bf}}{
	\renewcommand{\captiontextbold}{}
}
\ifthenelse{\equal{\captiontextsubnumbold}{true}}{ % Número en negritas
	\renewcommand{\captiontextsubnumbold}{bf}}{
	\renewcommand{\captiontextsubnumbold}{}
}

% Se configura el texto de los caption
\corecheckfontsize{\captionfontsize}
\captionsetup{
	font={\captionfontsize},
	labelfont={color=\captioncolor, \captiontextbold},
	labelformat={\captionlabelformat},
	labelsep={\captionlabelsep},
	textfont={color=\captiontextcolor},
	singlelinecheck=on
}

% Configura texto de los subcaption
\corecheckfontsize{\subcaptionfsize}
\captionsetup*[subfigure]{
	font={\subcaptionfsize},
	labelfont={color=\captioncolor, \captiontextsubnumbold},
	labelformat={\subcaptionlabelformat},
	labelsep={\subcaptionlabelsep},
	lofdepth=1,
	textfont={color=\captiontextcolor},
	singlelinecheck=on
}
\captionsetup*[subtable]{
	font={\subcaptionfsize},
	labelfont={color=\captioncolor, \captiontextsubnumbold},
	labelformat={\subcaptionlabelformat},
	labelsep={\subcaptionlabelsep},
	lofdepth=1,
	textfont={color=\captiontextcolor},
	singlelinecheck=on
}

\makeatletter
\renewcommand\p@subfigure{\thefigure\captionsubchar}
\renewcommand\p@subtable{\thetable\captionsubchar}
\makeatother

% Configuración de márgenes en las figuras
\floatsetup[figure]{
	captionskip=\captiontbmarginfigure pt
}

% Configuración de márgenes en las tablas
\floatsetup[table]{
	captionskip=\captiontbmargintable pt
}

% Caption superior en figuras
\ifthenelse{\equal{\figurecaptiontop}{true}}{
	\floatsetup[figure]{position=above}}{
}

% Caption superior en tablas
\ifthenelse{\equal{\tablecaptiontop}{true}}{
	\floatsetup[table]{position=top}
	}{
	\floatsetup[table]{position=bottom}
}

% Alineado de leyendas
\ifthenelse{\equal{\captionalignment}{justified}}{ % Leyenda justificada
	\captionsetup{
		format=plain,
		justification=justified
	}
}{
\ifthenelse{\equal{\captionalignment}{centered}}{ % Leyenda centrada
	\captionsetup{
		justification=centering
	}
}{
\ifthenelse{\equal{\captionalignment}{left}}{ % Leyenda alineada a la izquierda
	\captionsetup{
		justification=raggedright,
		singlelinecheck=false
	}
}{
\ifthenelse{\equal{\captionalignment}{right}}{ % Leyenda alineada a la derecha
	\captionsetup{
		justification=raggedleft,
		singlelinecheck=false
	}
}{
	\throwbadconfig{Posicion de leyendas desconocida}{\captionalignment}{justified,centered,left,right}}}}
}

% -----------------------------------------------------------------------------
% Configuración de referencias y citas
% -----------------------------------------------------------------------------
\ifthenelse{\equal{\stylecitereferences}{bibtex}}{
	\bibliographystyle{\bibtexstyle}
	\newlength{\bibitemsep}
	\setlength{\bibitemsep}{.2\baselineskip plus .05\baselineskip minus .05\baselineskip}
	\newlength{\bibparskip}\setlength{\bibparskip}{0pt}
	\let\oldthebibliography\thebibliography
	\renewcommand\thebibliography[1]{
		\oldthebibliography{#1}
		\setlength{\parskip}{\bibitemsep}
		\setlength{\itemsep}{\bibparskip}
	}
	\setlength{\bibitemsep}{\bibtexrefsep pt}
}{
\ifthenelse{\equal{\stylecitereferences}{custom}}{
	\coretemplatemessage{Usando estilo citas referencias custom, importar librerias y configuraciones posterior al llamado de template.tex en archivo principal}
}{
	\throwbadconfig{Estilo citas desconocido}{\stylecitereferences}{bibtex,custom}}
}

% -----------------------------------------------------------------------------
% Reconfiguración de tamaño de páginas
% -----------------------------------------------------------------------------
\makeatletter
	\def\ifGm@preamble#1{\@firstofone}
	\appto\restoregeometry{
		\pdfpagewidth=\paperwidth
		\pdfpageheight=\paperheight}
	\apptocmd\newgeometry{
		\pdfpagewidth=\paperwidth
		\pdfpageheight=\paperheight}{}{}
\makeatother

% -----------------------------------------------------------------------------
% Configuración de hbox y vbox
% -----------------------------------------------------------------------------
\hfuzz=200pt
\vfuzz=200pt
\hbadness=\maxdimen
\vbadness=\maxdimen

% -----------------------------------------------------------------------------
% Configura las fuentes
% -----------------------------------------------------------------------------
\makeatletter
\def\Hv@scale {.95}
\makeatother

% -----------------------------------------------------------------------------
% Configuraciones de las tablas
% -----------------------------------------------------------------------------
\makeatletter % Reinicia el número de cada fila en todas las tablas
\preto\tabular{\global\rownum=\z@}
\preto\tabularx{\global\rownum=\z@}
\makeatother

% -----------------------------------------------------------------------------
% Se activa el word-wrap para textos con \texttt{}
% -----------------------------------------------------------------------------
\ttfamily \hyphenchar\the\font=`\-

% -----------------------------------------------------------------------------
% Se define el tipo de texto de los url
% -----------------------------------------------------------------------------
\urlstyle{\fonturl}

% -----------------------------------------------------------------------------
% Configuraciones del motor de compilación
% -----------------------------------------------------------------------------
\ifthenelse{\equal{\compilertype}{pdf2latex}}{
	% Nivel de compresión
	\pdfcompresslevel=\pdfcompilecompression
	
	% El óptimo es 2, según
	% https://texdoc.org/serve/pdftex-a.pdf/0 p.20
	\pdfdecimaldigits=2
	
	% Inclusión de PDF
	\pdfinclusionerrorlevel=0
	
	% Compresión de objetos
	\pdfobjcompresslevel=\pdfcompileobjcompression
}{
\ifthenelse{\equal{\compilertype}{xelatex}}{
}{
\ifthenelse{\equal{\compilertype}{lualatex}}{
}{
	\throwbadconfig{Compilador desconocido}{\compilertype}{pdf2latex,xelatex,lualatex}}}
}

% -----------------------------------------------------------------------------
% Crea las sub-sub-sub-secciones
% -----------------------------------------------------------------------------
\newcounter{subsubsubsection}[subsubsection]

% Límite máximo profundidad
\setcounter{secnumdepth}{4}

% -----------------------------------------------------------------------------
% Profundidad del índice y bookmarks pdf
% -----------------------------------------------------------------------------
\setcounter{tocdepth}{\indexdepth}

% -----------------------------------------------------------------------------
% Configuración footnotes
% -----------------------------------------------------------------------------
% Restaura número
\ifthenelse{\equal{\footnoterestart}{none}}{
	% \counterwithout*{footnote}{chapter}
}{
\ifthenelse{\equal{\footnoterestart}{sec}}{
	\counterwithin*{footnote}{section}
}{
\ifthenelse{\equal{\footnoterestart}{ssec}}{
	\counterwithin*{footnote}{subsection}
}{
\ifthenelse{\equal{\footnoterestart}{sssec}}{
	\counterwithin*{footnote}{subsubsection}
}{
\ifthenelse{\equal{\footnoterestart}{ssssec}}{
	\counterwithin*{footnote}{subsubsubsection}
}{
\ifthenelse{\equal{\footnoterestart}{page}}{
	\counterwithin*{footnote}{page}
}{
\ifthenelse{\equal{\footnoterestart}{chap}}{
	\counterwithin*{footnote}{chapter}
}{
	\throwbadconfig{Formato reinicio numero footnote desconocido}{\footnoterestart}{none,chap,page,sec,ssec,sssec,ssssec}}}}}}}
}

% Define el tamaño del margen
\setlength{\footnotemargin}{\footnotelmargin pt}

% Previene footnote en otras páginas
\interfootnotelinepenalty=10000

% Configura tablas y figuras
\ifthenelse{\equal{\footnoterulefigure}{false}}{
	\floatsetup[figure]{footnoterule=none}}{
}
\ifthenelse{\equal{\footnoteruletable}{false}}{
	\floatsetup[table]{footnoterule=none}}{
}

% -----------------------------------------------------------------------------
% Restauración número ecuación, NOTA: NO hace nada, sólo se modifica en title.tex
% -----------------------------------------------------------------------------
\ifthenelse{\equal{\equationrestart}{none}}{
}{
\ifthenelse{\equal{\equationrestart}{chap}}{
}{
\ifthenelse{\equal{\equationrestart}{sec}}{
}{
\ifthenelse{\equal{\equationrestart}{ssec}}{
}{
\ifthenelse{\equal{\equationrestart}{sssec}}{
}{
\ifthenelse{\equal{\equationrestart}{ssssec}}{
}{
	\throwbadconfig{Formato reinicio numero ecuacion desconocido}{\equationrestart}{none,chap,sec,ssec,sssec,ssssec}}}}}}
}

% -----------------------------------------------------------------------------
% Configuración elementos matemáticos
% -----------------------------------------------------------------------------
\newtheoremstyle{templatetheorem}{\baselineskip}{3pt}{\itshape}{}{\bfseries}{}{.5em}{}
\newtheoremstyle{templateobs}{\baselineskip}{3pt}{}{}{\bfseries}{}{.5em}{}
\theoremstyle{templatetheorem}

% Configura números
\ifthenelse{\equal{\showsectioncaptionmat}{none}}{
	\newtheorem{defn}{\namemathdefn}
	\newtheorem{teo}{\namemaththeorem}
	\newtheorem{cor}{\namemathcol}
	\newtheorem{lema}{\namemathlem}
	\newtheorem{prop}{\namemathprp}
}{
\ifthenelse{\equal{\showsectioncaptionmat}{chap}}{
	\newtheorem{defn}{\namemathdefn}[chapter]
	\newtheorem{teo}{\namemaththeorem}[chapter]
	\newtheorem{cor}{\namemathcol}[chapter]
	\newtheorem{lema}{\namemathlem}[chapter]
	\newtheorem{prop}{\namemathprp}[chapter]
}{
\ifthenelse{\equal{\showsectioncaptionmat}{sec}}{
	\newtheorem{defn}{\namemathdefn}[section]
	\newtheorem{teo}{\namemaththeorem}[section]
	\newtheorem{cor}{\namemathcol}[section]
	\newtheorem{lema}{\namemathlem}[section]
	\newtheorem{prop}{\namemathprp}[section]
}{
\ifthenelse{\equal{\showsectioncaptionmat}{ssec}}{
	\newtheorem{defn}{\namemathdefn}[subsection]
	\newtheorem{teo}{\namemaththeorem}[subsection]
	\newtheorem{cor}{\namemathcol}[subsection]
	\newtheorem{lema}{\namemathlem}[subsection]
	\newtheorem{prop}{\namemathprp}[subsection]
}{
\ifthenelse{\equal{\showsectioncaptionmat}{sssec}}{
	\newtheorem{defn}{\namemathdefn}[subsubsection]
	\newtheorem{teo}{\namemaththeorem}[subsubsection]
	\newtheorem{cor}{\namemathcol}[subsubsection]
	\newtheorem{lema}{\namemathlem}[subsubsection]
	\newtheorem{prop}{\namemathprp}[subsubsection]
}{
\ifthenelse{\equal{\showsectioncaptionmat}{ssssec}}{
	\newtheorem{defn}{\namemathdefn}[subsubsubsection]
	\newtheorem{teo}{\namemaththeorem}[subsubsubsection]
	\newtheorem{cor}{\namemathcol}[subsubsubsection]
	\newtheorem{lema}{\namemathlem}[subsubsubsection]
	\newtheorem{prop}{\namemathprp}[subsubsubsection]
}{
	\throwbadconfig{Valor configuracion incorrecto}{\showsectioncaptionmat}{none,chap,sec,ssec,sssec,ssssec}}}}}}
}
\theoremstyle{templateobs}
\newtheorem*{ej}{\namemathej}
\newtheorem*{obs}{\namemathobs}

% -----------------------------------------------------------------------------
% Configura el formato oneside/twoside
% -----------------------------------------------------------------------------
% Normaliza el formato de páginas
\raggedbottom

% -----------------------------------------------------------------------------
% Configuraciones del idioma
% -----------------------------------------------------------------------------
% Desactiva caracteres acentuados en operaciones matemáticas
\unaccentedoperators

% -----------------------------------------------------------------------------
% Configura número de objetos en el final del documento
% -----------------------------------------------------------------------------
\AtEndDocument{
	\addtocounter{equation}{\value{templateEquations}}
	\addtocounter{figure}{\value{templateFigures}}
	\addtocounter{lstlisting}{\value{templateListings}}
	\addtocounter{table}{\value{templateTables}}
}

% -----------------------------------------------------------------------------
% Formato de columnas
% -----------------------------------------------------------------------------
% Centrado
\newcolumntype{C}[1]{>{\centering\let\newline\\\arraybackslash\hspace{0pt}}m{#1}}
\newcolumntype{\CColor}[2]{>{\columncolor{#1}\centering\let\newline\\\arraybackslash\hspace{0pt}}m{#2}}

\newcolumntype{P}[1]{>{\centering\let\newline\\\arraybackslash\hspace{0pt}}p{#1}}
\newcolumntype{\PColor}[2]{>{\columncolor{#1}\centering\let\newline\\\arraybackslash\hspace{0pt}}p{#2}}

\newcolumntype{B}[1]{>{\centering\let\newline\\\arraybackslash\hspace{0pt}}b{#1}}
\newcolumntype{\BColor}[2]{>{\columncolor{#1}\centering\let\newline\\\arraybackslash\hspace{0pt}}b{#2}}

% Izquierda
\newcolumntype{L}[1]{>{\raggedright\let\newline\\\arraybackslash\hspace{0pt}}m{#1}}
\newcolumntype{\LColor}[2]{>{\columncolor{#1}\raggedright\let\newline\\\arraybackslash\hspace{0pt}}m{#2}}
\newcolumntype{T}[1]{>{\raggedright\let\newline\\\arraybackslash\hspace{0pt}}p{#1}}
\newcolumntype{\TColor}[2]{>{\columncolor{#1}\raggedright\let\newline\\\arraybackslash\hspace{0pt}}p{#2}}
\newcolumntype{F}[1]{>{\raggedright\let\newline\\\arraybackslash\hspace{0pt}}b{#1}}
\newcolumntype{\FColor}[2]{>{\columncolor{#1}\raggedright\let\newline\\\arraybackslash\hspace{0pt}}b{#2}}

% Derecha
\newcolumntype{R}[1]{>{\raggedleft\let\newline\\\arraybackslash\hspace{0pt}}m{#1}}
\newcolumntype{\RColor}[2]{>{\columncolor{#1}\raggedleft\let\newline\\\arraybackslash\hspace{0pt}}m{#2}}
\newcolumntype{H}[1]{>{\raggedleft\let\newline\\\arraybackslash\hspace{0pt}}p{#1}}
\newcolumntype{\HColor}[2]{>{\columncolor{#1}\raggedleft\let\newline\\\arraybackslash\hspace{0pt}}p{#2}}
\newcolumntype{G}[1]{>{\raggedleft\let\newline\\\arraybackslash\hspace{0pt}}b{#1}}
\newcolumntype{\GColor}[2]{>{\columncolor{#1}\raggedleft\let\newline\\\arraybackslash\hspace{0pt}}b{#2}}

% -----------------------------------------------------------------------------
% Parcha el entorno tablenotes
% -----------------------------------------------------------------------------
\BeforeBeginEnvironment{tablenotes}{%
	\tablenotesfontsize\selectfont%
}
\AfterEndEnvironment{tablenotes}{%
	\normalsize\selectfont%
}

% -----------------------------------------------------------------------------
% Parcha el entorno multicols
% -----------------------------------------------------------------------------
\let\SOURCEcaptionlrmargin\captionlrmargin
\newcounter{multicoldepth}
\setcounter{multicoldepth}{0}
\BeforeBeginEnvironment{multicols}{%
	\def\captionlrmargin {\captionlrmarginmc}%
	\global\def\GLOBALenvmulticol {true}%
	\setcaptionmargincm{\captionlrmargin}%
	\addtocounter{multicoldepth}{1}%
}
\AfterEndEnvironment{multicols}{%
	\def\captionlrmargin {\SOURCEcaptionlrmargin}%
	\setcaptionmargincm{\captionlrmargin}%
	\addtocounter{multicoldepth}{-1}
	\ifnumequal{\number\value{multicoldepth}}{0}{%
		\global\def\GLOBALenvmulticol {false}
	}{}
}

% -----------------------------------------------------------------------------
% Configura estilos de listas
% -----------------------------------------------------------------------------
% Enumerate
\def\labelenumi {\textcolor{\enumerateitemcolor}{\senumerti}}
\def\labelenumii {\textcolor{\enumerateitemcolor}{\senumertii}}
\def\labelenumiii {\textcolor{\enumerateitemcolor}{\senumertiii}}
\def\labelenumiv {\textcolor{\enumerateitemcolor}{\senumertiv}}

% Itemize
\def\labelitemi {\textcolor{\itemizeitemcolor}{\sitemizei}}
\def\labelitemii {\textcolor{\itemizeitemcolor}{\sitemizeii}}
\def\labelitemiii {\textcolor{\itemizeitemcolor}{\sitemizeiii}}
\def\labelitemiv {\textcolor{\itemizeitemcolor}{\sitemizeiv}}

% Márgenes
\setlength\leftmargini{\sitemsmargini pt}
\setlength\leftmarginii{\sitemsmarginii pt}
\setlength\leftmarginiii{\sitemsmarginiii pt}
\setlength\leftmarginiv{\sitemsmarginiv pt}

% -----------------------------------------------------------------------------
% Chequea que ciertos módulos no hayan sido cargados antes del inicio del documento
% -----------------------------------------------------------------------------
\checkmodulenotloaded{tcolorbox}

% -----------------------------------------------------------------------------
% Da soporte a \hl{} del paquete soul
% -----------------------------------------------------------------------------
\soulregister\cite7
\soulregister\eqref7
\soulregister\eqref7
\soulregister\footnote7
\soulregister\href7
\soulregister\pageref7
\soulregister\quotes7
\soulregister\ref7
\soulregister\scite7

% -----------------------------------------------------------------------------
% Configura métodos aplicados al iniciar el documento
% -----------------------------------------------------------------------------
\AtBeginDocument{%
	\normalfont%
	\setlength{\parindent}{\documentparindent pt}%
	\setlength{\parskip}{\documentparskip pt}%
}

% -----------------------------------------------------------------------------
% Justificación de textos
% -----------------------------------------------------------------------------
\ifthenelse{\equal{\frametextjustified}{true}}{
	\apptocmd{\frame}{}{\justifying}{}
}{}
\newcommand{\justifytext}[1]{\parbox{\linewidth}{#1}}

% -----------------------------------------------------------------------------
% Word-break en citas
% -----------------------------------------------------------------------------
\makeatletter
\let\@cite@ofmt\@firstofone
\makeatother

% -----------------------------------------------------------------------------
% Corrige espaciamiento de itemize
% -----------------------------------------------------------------------------
\ifthenelse{\equal{\itemizedeleteleftmargin}{true}}{
	\setlist[itemize]{leftmargin=*}
	\setlist[enumerate]{leftmargin=*}
}{}

% -----------------------------------------------------------------------------
% Cambios generales en presentación
% -----------------------------------------------------------------------------
% \let\Tiny=\tiny % https://tex.stackexchange.com/q/58087/5764
% Desactiva color links en headline y footline
\makeatletter
	\renewcommand\insertshorttitle[1][]{%
		\beamer@setupshort{#1}%
		\let\thanks=\@gobble%
		\ifnum\c@page=1%
		\hyperlinkpresentationend{\beamer@insertshort{\usebeamercolor*[fg]{title in head/foot}\beamer@shorttitle}}%
		\else%
		\hyperlinkpresentationstart{\beamer@insertshort{\usebeamercolor*[fg]{title in head/foot}\beamer@shorttitle}}%
		\fi%
	}
	\newcommand\disablebeamercolorlinks{\def\HyColor@UseColor##1{}}
\makeatletter
\addtobeamertemplate{headline}{\disablebeamercolorlinks{}}{}
\addtobeamertemplate{footline}{\disablebeamercolorlinks{}}{}
% Modifica color de fondo de la página
\setbeamercolor{background canvas}{bg=\pagescolor}

% -----------------------------------------------------------------------------
% Configura los bloques
% -----------------------------------------------------------------------------
% Normal
\addtobeamertemplate{block begin}{}{
	\setlength\abovedisplayskip{0pt}%
	\vspace{\dimexpr-0.7em + \blockpaddingtop em} % Padding superior bloque
	\begin{adjustwidth}{%
		\dimexpr-0.4em + \blockpaddingleft em % Padding izquierdo bloque
	}{
		\dimexpr-0.4em + \blockpaddingright em % Padding derecho bloque
	}
}
\addtobeamertemplate{block end}{
	\end{adjustwidth}
	\vspace{\dimexpr-0.6em + \blockpaddingbottom em} % Padding inferior bloque
}{
	\vspace{\dimexpr-1.07em + \blockmarginbottom em} % Margen inferior bloque
}
% Alerta
\addtobeamertemplate{block alerted begin}{}{
	\setlength\abovedisplayskip{0pt}%
	\vspace{\dimexpr-0.7em + \blockpaddingtop em} % Padding superior alerta
	\begin{adjustwidth}{%
			\dimexpr-0.4em + \blockpaddingleft em % Padding izquierdo alerta
		}{
			\dimexpr-0.4em + \blockpaddingright em % Padding derecho alerta
		}
	}
\addtobeamertemplate{block alerted end}{
	\end{adjustwidth}
	\vspace{\dimexpr-0.6em + \blockpaddingbottom em} % Padding inferior alerta
}{
	\vspace{\dimexpr-1.07em + \blockmarginbottom em} % Margen inferior alerta
}
% Ejemplo
\addtobeamertemplate{block example begin}{}{
	\setlength\abovedisplayskip{0pt}%
	\vspace{\dimexpr-0.7em + \blockpaddingtop em} % Padding superior ejemplo
	\begin{adjustwidth}{%
			\dimexpr-0.4em + \blockpaddingleft em % Padding izquierdo ejemplo
		}{
			\dimexpr-0.4em + \blockpaddingright em % Padding derecho ejemplo
		}
	}
\addtobeamertemplate{block example end}{
	\end{adjustwidth}
	\vspace{\dimexpr-0.6em + \blockpaddingbottom em} % Padding inferior ejemplo
}{
	\vspace{\dimexpr-1.07em + \blockmarginbottom em} % Margen inferior ejemplo
}

% -----------------------------------------------------------------------------
% Definición de entornos beamer
% -----------------------------------------------------------------------------
\newenvironment<>{blockjustified}[1]{%
	\begin{block}#2{#1}\justifying
}{%
	\end{block}
}
\newenvironment<>{alertblockjustified}[1]{%
	\begin{alertblock}#2{#1}\justifying
}{%
	\end{alertblock}
}
\newenvironment<>{exampleblockjustified}[1]{%
	\begin{exampleblock}#2{#1}\justifying
}{%
	\end{exampleblock}
}

% -----------------------------------------------------------------------------
% Reinicia número de subfiguras y subtablas
% -----------------------------------------------------------------------------
\makeatletter
	\@addtoreset{subfigure}{framenumber}
	\@addtoreset{subtable}{framenumber}
\makeatother

% -----------------------------------------------------------------------------
% Establece temas custom
% -----------------------------------------------------------------------------
\makeatletter
\ifthenelse{\equal{\GLOBALcustombeamertheme}{}}{%
}{
\ifthenelse{\equal{\GLOBALcustombeamertheme}{FrankfurtBottom}}{%
	\usetheme{Frankfurt}%
	\setbeamertemplate{frametitle}[default][colsep=-4bp,rounded=false,shadow=false]%
	\setbeamertemplate{headline}{}%
	\setbeamertemplate{footline}{%
		\hypersetup{hidelinks}%
		\pgfuseshading{beamer@barshade}%
		\vskip-0.05ex%
		\begin{beamercolorbox}[wd=\paperwidth,ht=1ex,dp=0ex]{empty}
			\begin{pgfpicture}
				\begin{pgfscope}%
					\ifbeamer@sb@subsection%
					\pgfsetfillcolor{subsection in head/foot.bg}%
					\else%
					\pgfsetfillcolor{section in head/foot.bg}%
					\fi%
					\pgfpathrectangle{\pgfpoint{-.5\paperwidth}{-0.5ex}}{\pgfpoint{\paperwidth}{1ex}}%
					\pgfsetfading{beamer@belowframetitlemask}{}%
					\pgfusepath{fill}%
				\end{pgfscope}%
			\end{pgfpicture}%
		\end{beamercolorbox}%
		\ifbeamer@sb@subsection%
		\vskip-9.75ex%
		\else%
		\vskip-6.85ex%
		\fi%
		\begin{beamercolorbox}[ignorebg,ht=2.25ex,dp=3.75ex]{section in head/foot}
			\insertnavigation{\paperwidth}
		\end{beamercolorbox}%
		\ifbeamer@sb@subsection%
		\begin{beamercolorbox}[ignorebg,ht=2.125ex,dp=1.125ex,%
			leftskip=.3cm,rightskip=.3cm plus1fil]{subsection in head/foot}
			\usebeamerfont{subsection in head/foot}\insertsubsectionhead
		\end{beamercolorbox}%
		\fi%
	}
}{
	\errmessage{Tema no existe}
	\stop
}}
\makeatother

% -----------------------------------------------------------------------------
% Configura footnotes
% -----------------------------------------------------------------------------
\let\oldfootnote\footnote
\renewcommand\footnote[1][]{\oldfootnote[frame,#1]}
