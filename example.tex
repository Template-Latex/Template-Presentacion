% Template:     Template Presentación LaTeX
% Documento:    Archivo de ejemplo
% Versión:      1.1.3 (21/06/2021)
% Codificación: UTF-8
%
% Autor: Pablo Pizarro R.
%        Facultad de Ciencias Físicas y Matemáticas.
%        Universidad de Chile.
%        pablo@ppizarror.com
%
% Sitio web:    [https://latex.ppizarror.com/presentacion]
% Licencia MIT: [https://opensource.org/licenses/MIT]

% Portada
\begin{frame}
	\titlepage
\end{frame}

% Tabla de contenidos
\begin{frame}
	\frametitle{Contenidos}
	\tableofcontents
\end{frame}

% Organiza el documento en secciones, útiles para la tabla de contenidos
\section{Primera sección}

\subsection{Ejemplo de sub-sección}

%---------------------------------------------------------------------

\begin{frame}
	\frametitle{Empezando con el mejor template}
	\lipsum[1]
\end{frame}

%---------------------------------------------------------------------

\begin{frame}
	\frametitle{Enumeraciones}
	\begin{itemize}
		\item Lorem ipsum dolor sit amet, consectetur adipiscing elit
		\item Aliquam blandit faucibus nisi, sit amet dapibus enim tempus eu
		\item Nulla commodo, erat quis gravida posuere, elit lacus lobortis est, quis porttitor odio mauris at libero
		\item Nam cursus est eget velit posuere pellentesque
		\item Vestibulum faucibus velit a augue condimentum quis convallis nulla gravida
	\end{itemize}
\end{frame}

%---------------------------------------------------------------------

\begin{frame}
	\frametitle{Bloques de texto}
	\begin{block}{Bloque 1 (justificado)}
		\justifytext{Lorem ipsum dolor sit amet, consectetur adipiscing elit. Integer lectus nisl, ultricies in feugiat rutrum, porttitor sit amet augue. Aliquam ut tortor mauris. Sed volutpat ante purus, quis accumsan dolor.}
	\end{block}
	
	\begin{block}{Bloque 2}
		Pellentesque sed tellus purus. Class aptent taciti sociosqu ad litora torquent per conubia nostra, per inceptos himenaeos. Vestibulum quis magna at risus dictum tempor eu vitae velit.
	\end{block}
	
	\begin{block}{Bloque 3}
		Suspendisse tincidunt sagittis gravida. Curabitur condimentum, enim sed venenatis rutrum, ipsum neque consectetur orci, sed blandit justo nisi ac lacus.
	\end{block}
\end{frame}

%---------------------------------------------------------------------

\begin{frame}
	\frametitle{Múltiples columnas}
	\begin{columns}[c]
		
		\column{.45\textwidth} % Columna izquierda y espesor
		\textbf{Heading}
		\begin{enumerate}
			\item Statement
			\item Explanation
			\item Example
		\end{enumerate}
		
		\column{.5\textwidth} % Columna derecha y espesor
		Lorem ipsum dolor sit amet, consectetur adipiscing elit. Integer lectus nisl, ultricies in feugiat rutrum, porttitor sit amet augue. Aliquam ut tortor mauris. Sed volutpat ante purus, quis accumsan dolor.
		
	\end{columns}
\end{frame}

\section{Segunda sección}

%---------------------------------------------------------------------

\begin{frame}
	\frametitle{Tabla}
	\begin{table}
		\centering
		\caption{Mi tabla.}
		\begin{tabular}{ccc}
			\hline
			\textbf{Columna 1} & \textbf{Columna 2} & \textbf{Columna 3} \bigstrut\\
			\hline
			$\omega$ & $\nu$ & $\delta$ \bigstrut[t]\\
			$\Phi$ & $\Theta$ & $\varSigma$ \\
			$\xi$ & $\kappa$ & $\varpi$ \bigstrut[b] \\
			\hline
		\end{tabular}
	\end{table}
\end{frame}

%---------------------------------------------------------------------

\begin{frame}[fragile]
	\frametitle{Código fuente (Template)}
	Al igual que \href{https://latex.ppizarror.com/informe}{Template-Informe}, Template-Presentación ofrece el soporte a todos los lenguajes de programación con el entorno \texttt{sourcecode}.
\begin{sourcecode}{python}{Ejemplo en Python.}
import numpy as np

def incmatrix(genl1, genl2):
	m = len(genl1)
	n = len(genl2)
	M = None # Comentario 1
	VT = np.zeros((n*m, 1), int) # Comentario 2
\end{sourcecode}
\end{frame}

%---------------------------------------------------------------------

\begin{frame}[fragile]
	\frametitle{Código fuente (Verbatim)}
	\begin{example}[Código sencillo en Python]
		\begin{verbatim}
		import numpy as np

		def incmatrix(genl1, genl2):
			m = len(genl1)
			n = len(genl2)
			M = None # Comentario 1
			VT = np.zeros((n*m, 1), int) # Comentario 2
		\end{verbatim}
	\end{example}
\end{frame}

%---------------------------------------------------------------------

\begin{frame}
	\frametitle{Figuras}
	La Figura \ref{img:testimage} ilustra una imagen insertada con las funciones propias del template, mismas compartidas con todos los subtemplates.
	
	\insertimage[\label{img:testimage}]{ejemplos/test-image.png}{scale=0.14}{Where are you? de \quotes{Internet}.}
\end{frame}

%---------------------------------------------------------------------

\begin{frame}
	\frametitle{Figuras múltiples}
	\begin{columns}[c]
		
		\column{.65\textwidth} % Columna izquierda y espesor
		\begin{images}{Ejemplo de imagen múltiple.}
			\addimage{ejemplos/test-image}{width=3cm}{Ciudad}
			\addimageanum{ejemplos/test-image-wrap}{height=2cm}
			\imagesnewline
			\addimage{ejemplos/test-image}{width=4cm}{Ciudad más grande}
		\end{images}
	
		\column{.35\textwidth}
		
		Enumerar con letras:
		
		\begin{enumeratebf}[label=\alph*) ] % Fuente en negrita
			\item Peras
			\item Manzanas
		\end{enumeratebf}
		
		O con números romanos:
		
		\begin{enumerate}[label=\roman*) ]
			\item Rojo
			\item Café
		\end{enumerate}
	
		$\ldots$ o griegos:
		
		\begin{enumerate}[label=\greek*) ]
			\item Matemáticas
			\item Lenguaje
		\end{enumerate}
	
	\end{columns}
\end{frame}

%---------------------------------------------------------------------

\begin{frame}
	\frametitle{Citas \& Ecuaciones}
	El template también ofrece opciones para citar \cite{templateinforme}. Análogamente, todas las fórmulas de ecuaciones de \href{https://latex.ppizarror.com/informe}{Template-Informe} son soportadas \eqref{eqn:eqn-larga}:
	
	\insertgathered[\label{eqn:eqn-larga}]{
		\lpow{\Lambda}{f} = \frac{L\cdot f}{W} \cdot \frac{\pow{\lpow{Q}{e}}{2}}{8 \pow{\pi}{2} \pow{W}{4} g} + \sum_{i=1}^{l} \frac{f \cdot \bigp{M - d}}{l \cdot W} \cdot \frac{\pow{\bigp{\lpow{Q}{e}- i\cdot Q}}{2}}{8 \pow{\pi}{2} \pow{W}{4} g}\\
		Q_e = 2.5Q \cdot \int_{0}^{e} V(x) \dd{x} + \aasin{\biggp{1+\frac{1}{1-e}}}
	}
\end{frame}

%---------------------------------------------------------------------

\begin{frame}
	\frametitle{Teoremas}
	\begin{theorem}[Equivalencia masa energía]
		$E = mc^2$
	\end{theorem}
\end{frame}

%---------------------------------------------------------------------

\begin{frame}
	\frametitle{Referencias}
	\footnotesize{
		\begin{thebibliography}{99}
			\bibitem{templateinforme}
			Template Informe en \LaTeX.
			\textit{¡Revisa el manual online de este template!} \\
			\url{https://latex.ppizarror.com/informe}
			
			\bibitem{excel2latex}
			Excel2Latex.
			\textit{Importa de forma sencilla tus tablas de Excel a \LaTeX.} \\
			\url{https://www.ctan.org/tex-archive/support/excel2latex/}
			
			\bibitem{overleaf}
			Overleaf.
			\textit{Uno de los mejores editores online para \LaTeX, renovado con su versión 2.0.} \\
			\href{https://www.overleaf.com?r=298b935f&rm=d&rs=b}{\hreftext{https://es.overleaf.com/}}
			
			\bibitem{tablesgenerator}
			Tables Generator.
			\textit{Creador de tablas online para \LaTeX.}\\
			\url{https://www.tablesgenerator.com}
		\end{thebibliography}
	}
\end{frame}

%---------------------------------------------------------------------

\begin{frame}
	\Huge{\centerline{Gracias por su atención}}
\end{frame}
